\documentclass{standalone}
\begin{document}
\chapter{Introduction}
\fancyhead[RE]{Introduction} % Chapter name on the right
A prevalent object in differential geometry is that of a surjective submersion, may it be as a vector bundle, principal \(G\)-bundle, covering space, associated bundle or symplectic fibration. A point of view on these subjects is through the foliation of the total space into the fibres of the map. However, the nature of such foliations is rather tame as the only interesting geometry lies transversal to the leaves. Therefore, we are interested in the interplay between the geometry of the domain and codomain. In many geometrical theories using surjective submersion, like the ones mentioned before, there are additional homogeneity conditions imposed on the surjective submersion such that it locally resembles a product space. Such structures are known as fibre bundles, and they have been studied extensively as they give a strong relation between the domain \textemdash or total space \textemdash and the codomain \textemdash or base space \textemdash of the surjective submersion. For example, unlike general surjective submersions, a fibre bundle is a Serre fibration, cf.\ \cite{Husemoeller1994}.

In many of the aforementioned examples \textemdash in particular, vector bundles, principal \(G\)-bundles, covering spaces and symplectic fibrations \textemdash an important aspect of the theories deals with the lifting of paths from the base space to the total space or parallel transport of points along them. Integral to these types of problems is a notion of parallelness or horizontality, which is introduced through the concept of a connection. While this notion may differ between these fields, from affine connections to connection \(1\)-forms, they are all manifestations of (Ehresmann) connections on a surjective submersion, satisfying some compatibility conditions. An Ehresmann connection corresponds to a specific subbundle \(\Hor\subset TM\), for \(\pi\colon M\to B\), which is a complement to \(\ker T\pi\), and they were introduced by Charles Ehresmann \cite{Ehresmann1959}. From the basic theory of vector bundles, it follows that such a connection always exists; therefore, they can be used as a standard tool in the theory of surjective submersions.

Given an Ehresmann connection \(\Hor\) on a surjecrtive submersion \(\pi\colon M\to B\), some curve \(\gamma\h{0,1}\to B\) and a lift \(x\in \pi^{-1}\h{\gamma\h{0}}\), we can parallel transport \(x\) along \(\gamma\) by solving the following initial value problem:
\[\begin{cases}
	\tilde{\gamma}\h{0} = x,\\
	\dot{\tilde{\gamma}}\h{t}\in E_{\gamma\h{t}}.
\end{cases}\]
Generally, a solution is only local; when it always extends to the whole of \(\ha{0,1}\), a connection is called complete. In some cases \textemdash e.g.\ vector bundles, principal \(G\)-bundles and covering spaces \textemdash any connection with the correct compatibility conditions is complete. While the completeness of a connection is an analytical condition, these previous examples already show that geometric properties of a surjective submersion can ensure its existence.

In this thesis, we are interested in investigating such relations between the geometry imposed by the surjective submersion and the analytical properties of the connection. One of the main results of this thesis is the following:
\begin{theorem*}
	A surjective submersion admits a complete connection if and only if it is a fibre bundle.
\end{theorem*}
The idea of our proof is based on \cite{delHoyo2016}; however, we have reworked and generalised many constructions in the proof to give a better overview of the objects in the construction. The preceding theory to this result lets us generalise to a multiplicative version as well.\\

For the multiplicative version, we are interested in another recurring topic within differential geometry: that of Lie groupoids. They were first introduced to study generalised symmetries in the 1950s by Charles Ehresmann \cite{Ehresmann1959} and were thoroughly investigated by his PhD students. However, they became mainstream mathematical objects due to two significant applications. Firstly, Alain Connes stressed their importance in his theory of noncommutative geometry, e.g. \cite{Connes1990}. Secondly, they are used to ``integrate'' Poisson structures, as introduced by Alan Weinstein in \cite{Weinstein1987}.

Recent developments surrounding different normal form theorems have sparked particular interest in types of surjective submersions by Lie groupoids morphisms. For example, in the deformation theory of Lie groupoids and related structures, like symplectic Lie groupoids, one considers Lie groupoid morphisms mapping onto an identity Lie groupoid which are surjective submersions, cf.\ \cite{Crainic2018, Cardenas2021}. Alternatively, one can consider the groupoid generalisation of a group extension, which is a short exact sequence of groups, as discussed in \cite{LaurentGengoux2009}. In the current literature on this topic, for example \cite{Fernandes2023}, a theory of Lie groupoid extensions using multiplicative Ehresmann connections has been developed in the case where the Lie groupoids are all over the same base space and the morphism covers the identity.

To provide a unifying framework for both these situations, we consider Lie groupoid morphisms, which may not cover the identity and which may not map to the identity groupoid, but which are surjective submersions. Such structures, we will call a \df{fibred Lie groupoid}.

While fibred Lie groupoids are a generalisation of the notions above, one of them still plays an integral part in the theory: Families of Lie groupoids. Inspired by the approach in \cite{Fernandes2023},  part of the geometry of a fibred Lie groupoid can be reduced to an internal family of Lie groupoids. In a traditional Lie groupoid extension, the kernel of the morphisms defines a bundle of Lie groups; however, in our generalised setting, we obtain a family of Lie groupoids instead. This family of Lie groupoids will also play an important role in the main results of the thesis, as they admit a notion of local triviality where the multiplicative structure of the fibres is incorporated.\\

Much like for many theories involving surjective submersions, e.g., vector bundles, principal \(G\)-bundles, it is fruitful to consider connections with certain compatibility conditions. For a fibred Lie groupoid, this compatibility comes from the multiplicative structure on the tangent bundle of a Lie groupoid. This compatibility can be presented in terms of the lifting of multiplicable curves, which gives a geometric interpretation akin to other classical theories of connections. Connections satisfying these conditions are called \df{multiplicative}.

We would like to emulate the above theorem on surjective submersion in the case of fibred Lie groupoids, as this has already been done for Lie groupoid extensions, see \cite{Fernandes2023}. However, due to problems with local triviality, we can only formulate this for families of Lie groupoids. One of the directions of the previous theorem translates directly, namely, complete connections giving local triviality. For the other direction, we have a problem of glueing multiplicative connections, and in particular, the problem of the existence of connections. Again, in the special case of Lie groupoid extensions, the existence of multiplicative connections is well-known and controlled by a class in cohomology \cite{Grad2025, LaurentGengoux2009}. Under additional compactness and local triviality assumptions, we can ensure the existence of a multiplicative connection and its completeness as well.
\begin{theorem*}
	Let \(p\colon\grK\to B\) be a locally trivial family of Lie groupoids with typical fibre \(\grG\). Suppose that \(\grG\) is a Lie groupoid whose source map is proper, then \(p\) admits a complete multiplicative connection.
\end{theorem*}
Additionally, we can reduce the completeness of multiplicative connections on arbitrary fibred Lie groupoids to the underlying kernel. A complete connection \(\Hor\) on a fibred Lie groupoid \(\phi\colon\grG\to\grH\) immediately defines a complete connection on its kernel given by \(\Hor^{\grK} = \Hor\cap T\ker\phi\). The converse of this can be shown to hold in the case where our morphism admits lifts to arbitrary sources, something we will call \df{arrow complete}.
\begin{theorem*}
	If \(\phi\colon\grG\to\grH\) is a fibred Lie groupoid that is arrow complete and a connection \(\Hor\) such that \(\Hor^{\grK}\) is complete, then \(\Hor\) is complete.
\end{theorem*}
Besides this application of arrow completeness, we will show that it automatically induces some equivalence between fibres of a fibred Lie groupoid, namely, Morita equivalence, even without the presence of a connection.\\

Lastly, this thesis discusses the notion of a symplectic Lie groupoid fibration, which incorporates the multiplicative structure of a fibred Lie groupoid such that it naturally combines with the fibred structure of a symplectic fibration. First, we give a digression on the uses of connections in symplectic fibrations, and in particular, we discuss why our proofs relating completeness to local triviality fail for a symplectic setting. We then show that symplectic Lie groupoid fibrations give a natural setting to translate classical results on symplectic fibrations to a multiplicative setting. Additionally, these types of structures seem to play a role in the theory of normal forms around Poisson submanifolds \cite{Fernandes2024}.

\section*{Organisation}
This thesis is organised as follows:
\begin{itemize}
	\item Chapter 1 concerns itself with the classical case of surjective submersions and connections. While many proofs in this chapter are omitted, the last section gives a full and new proof of the main theorem, namely, the equivalence between surjective submersions with complete connections and fibre bundles.
	\item Chapter 2 describes the notion of a Lie groupoid, alongside some of the basic theory and constructions surrounding them. Secondly, we describe the notion of a Morita equivalence between Lie groupoids using principal bibundles.
	\item Chapter 3 gives a short overview of the theory of \VB-groupoids and multiplicative differential forms. Additionally, we show some new results relating to short exact sequences of \VB-groupoids.
	\item Chapter 4 defines the notion of fibred Lie groupoids and multiplicative connections on them, and in particular also families of Lie groupoids. We then prove some results regarding the completeness of multiplicative connections, relating them to local triviality conditions.
	\item Chapter 5 is a digression on the application of connections in the field of symplectic fibrations. Additionally, we provide a brief introduction to a possible multiplicative point of view on this topic, which incorporates the theory of multiplicative connections.
\end{itemize}
\end{document}