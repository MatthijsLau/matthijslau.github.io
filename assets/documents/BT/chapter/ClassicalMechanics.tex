\documentclass[class = report, crop = false]{standalone}
\usepackage{standalone}

\begin{document}
\chapter{Classical Mechanics as a Physicist}
In the previous chapters, we discussed abstract geometrical objects called symplectic manifolds. From this point forward, we will turn towards an application: classical physics. Using symplectic geometry we can build a formal theory of classical mechanics, namely Hamiltonian mechanics. As many mathematicians may not be familiar with the contemporary formulations of classical mechanics past Newton's formalism, we will first dedicate a chapter to introducing physics to showcase the heuristic approach and create some intuition behind the methods. To do this we will discuss three formalisms: Newtonian, Lagrangian and Hamiltonian. Each of these formalisms is based on different principles, resulting in different methods of solving mechanical systems. For anyone familiar with these descriptions of physics, this chapter can easily be skipped without any continuity problems. This chapter is a combination of \cite{Taylor2005} and \cite{Arnold2006}. The translation of classical mechanics to symplectic geometry will be dealt with in Chapter~\ref{chp: hamiltonian systems}.
\documentclass[class = article, crop = false]{standalone}
\usepackage{standalone}

\begin{document}
	\section{Newtonian Formalism}
	We will make our first step into describing classical mechanics by taking a look at Newton's formulation and basic principles in the form of his three laws of motion. Nowadays these still form the foundation of classical mechanics. Newton was one of the first persons who saw that we could describe physics using some general mathematical model, which can be seen as the goal of classical physics nowadays: to predict the motions of a physical space using a mathematical model. Before we go into the actual model Newton built, we will discuss how we can even translate a physical space to a mathematical one in the first place.
	
	\subsection{Space, Time and Kinematics}
	As the goal is to describe the motions of mechanical systems mathematically, we should first determine the types of systems we are studying and how we could translate these to mathematical spaces. The assumptions in classical mechanics are that the objects are relatively large such that there are no quantum mechanical effects and the speeds are relatively small to stay non-relativistic. We then follow our intuition and suppose that positions in physical space are points of a three-dimensional Euclidean space \(E^3\). We would like to induce some vector space structure onto \(E^3\), which can be done by fixing an origin \(o\in E^3\), also called an observer, and we then identify a point \(s\in E^3\) with the vector \(\vec{os}\in\R^3\). When working with these vectors, we then also need to choose some basis vectors. We have some freedom of choice for what position acts as the origin and how we arrange the basis vectors, depending on this choice the position of an object may seemingly change, see Example~\ref{exp: reference frame plane}. In a physical problem, we can often choose a reference frame that uses the symmetries of a system.
	\begin{example}\label{exp: reference frame plane}
		Let us assume that the reference frames are chosen such that the positions in this example are constrained to \(\R[2]\times\hv{0}\subset\R[3]\). Therefore, we will only consider the position on a plane instead of a three-dimensional space. Suppose that we have a ball at a point \(p\) on \(E^3\) and some observers \(A\) and \(B\), see Figure~\ref{fig: configuration}. We can then view the position \(p\) from both the reference frame of \(A\) and \(B\). We can then measure the position of \(P\) in both reference frames, see Figure~\ref{fig: measurements}. Observer \(A\) measures the position of \(P\) as \(r_{PA} = \h{4,-2}^T\), while Observer \(B\) measures \(P\) to be at \(r_{PB} = \h{-1,2}^T\).
	\end{example}
	\begin{figure}
		\centering
		\begin{subfigure}[t]{.49\textwidth}
			\centering
			\includegraphics{img/ReferenceFrame_Configuration.pdf}
			\caption{The space \(E^3\) given by some grid on a plane on which a ball is placed at position \(P\) and two observers are positioned at \(A\) and \(B\).}
			\label{fig: configuration}
		\end{subfigure}
		\hfill
		\begin{subfigure}[t]{.49\textwidth}
			\centering
			\includegraphics{img/ReferenceFrame_Measurement.pdf}
			\caption{The reference frames and measurements of observer \(A\) and \(B\). The reference frame of observer \(A\) is given in blue, and the measured position of \(P\) is denoted by \(r_{PA}\). For observer \(B\) everything is in red and the position is given by \(r_{PB}\).}
			\label{fig: measurements}
		\end{subfigure}
		\caption{Example of how to translate a physical situation, Figure~\ref{fig: configuration}, to measurements in different reference frames, Figure~\ref{fig: measurements}.}
		\label{fig: numberline}
	\end{figure}
	As we mentioned, we are not dealing with any relativistic effects in this theory, and hence, we can identify time as a separate axis, called the time axis identifiable with \(\R\). This time axis is important when talking about motion, as this is all about the position over time. We define the motion of an object as some smooth function \(\mathcal{C}:I\to E^3\), where \(I\) is an interval on the time axis. In our reference frame, we can obtain a function \(r:I\to\R[3]\) called the trajectory of the object. Notice that this is the composition of the actual motion with our choice of reference frame, hence, it is dependent on this choice which may differ over time. Using this mathematical trajectory, we can describe some physical quantities as vectors. Namely, the velocity and acceleration, \(v\) and \(a\) respectively, are defined as follows
	\begin{equation*}
		v\h{t} = \dv{t}r\h{t} = \dot{r}\h{t}\qquad\mbox{and}\qquad a\h{t} = \dv[2]{t}r\h{t} = \ddot{r}\h{t} = \dot{v}\h{t}.
	\end{equation*}
	The importance of these quantities stems from Newton's laws of motion. However, before discussing these in detail, we should find a way to model more than just a single object in a system. We have thus far described a position of a single object while we are often dealing with a system that includes many bodies that are interacting. Intuitively we assign a single \(E^3\) for each object in the system, which can be formalised in terms of a product space. In other words, in an \(n\)-body system a position of the system is described as a point in \(E^{3n} = E^3\times\cdots\times E^3\). The position of the bodies in the system is described using a single vector \(r = \h{r_1,\ldots, r_n}\in\R[3n]\), implying that the motion becomes a smooth map \(r\h{t} = \h{r_1\h{t},\ldots,r_n\h{t}}\in\R[3n]\). The definition of velocity and acceleration then still applies.
	
	\subsection{Newton's Laws of Motion}\label{sec: laws of motion}
	Newton's laws of motion determine the motions of a system through two important concepts: momentum and forces. Momentum is a quantity of an object and is given for some object of mass \(m\) by \(p = mv\). This is often intuitively thought of as the amount of movement an object has. Forces on the other hand should be thought of as the interactions bodies have with each other or the system. These come in many shapes and forms, for example, the gravitational force or Coulomb force. All these forces are the product of some interaction between two physical bodies and are described as some vector \(F\in\R[3]\) acting on a body. As we will approximate motion to our best capability, it will often be useful to neglect some forces acting on larger bodies, this choice will make more sense in the light of Newton's second law. Let us now state these laws.
	\begin{enumerate}[label = {\arabic*.}]
		\item If the sum of all forces acting on a body is zero, there exists a reference frame in which its velocity is constant.
		\item In any inertial frame, the time derivative of the momentum of a body is equal to the sum of forces acting on it.
		\item For every action, there is an equal and opposite reaction.
	\end{enumerate}
	While the first law seems to follow from the second law, we still need to ensure the existence of an inertial frame to use the second law. The second law then gives us the following equation, also called the equation of motion
	\begin{equation*}
		\sum_iF_i\h{r,\dot{r},t} = \dot{p}\h{t},
	\end{equation*}
	where \(F_i\) are all the forces acting on the body, \(p\h{t}\) is its momentum at time \(t\). We will often denote the sum of all forces acting on a body with \(F_{\net}\). If we also assume that the body has a constant mass this equation simplifies to
	\begin{equation*}
		F_{\net}\h{t,r,\dot{r}} = m\ddot{r}.
	\end{equation*}
	If we generalise this to an \(n\)-body system, we obtain an equation of motion for each of the bodies. In the case that the masses of all objects are constant, which is nearly always the case, this results in the following system of equations:
	\begin{equation}\label{eq: n equation of motion}
		\mqty(\dot{r}\\\dot{v}) = \mqty(v\\M^{-1}F\h{r,v,t}),\qquad M = \diag\h{m_1I_3,\ldots,m_nI_3}.
	\end{equation}
	Remark that we have transformed the differential equation by substituting \(v = \dot{r}\). This system of equations is the one we solve most often in Newtonian mechanics. The process of solving for the motions of a system uses the following steps:
	\begin{enumerate}[label = {\alph*)}]
		\item Choose a reference frame, and initial values then determine which forces are at play.
		\item Describe the forces in terms of the reference frame.
		\item Solve the equation of motion.
	\end{enumerate}
	Let us showcase this with a couple of examples.
	\begin{example}\label{exp: projectile}
		\documentclass[class = article, crop = false]{standalone}
\usepackage{standalone}

\begin{document}
	\begin{figure}
		\centering
		\includegraphics{img/Projectile_Sketch.pdf}
		\caption{A sketch of a cannon atop a hill shooting a cannonball at some angle and speed. In the figure, we plotted a trajectory which we calculated by numerically solving Newton's equations of a projectile motion in a constant gravitational field with linear and quadratic air resistance using the SciPy package in Python. Our initial velocity for this trajectory was set to \(15\) ms\(^{-1}\) in the \(x\)-direction and \(3\) ms\(^{-s}\) in the \(y\)-direction and the height of the cliff was set at \(10\) m. We will see in our analysis that the mass of the cannonball is irrelevant to the problem. At the foot of the cliff, the reference frame for the analysis is drawn as well.}
		\label{fig: sketch projectile}
	\end{figure}
	Suppose we have set up a cannon atop an \(h\) meter high cliff and we want to determine the distance it can shoot a cannonball which weighs \(m\) kilograms, see Figure~\ref{fig: sketch projectile}. We would then want to find a mathematical model to capture the motion of the cannonball. 
	
	First, we need to determine a reference frame which we have already chosen in Figure~\ref{fig: sketch projectile} to be at the foot of the cliff with the \(x\) corresponding to the horizontal direction and the \(y\) axis with the vertical one. Next up, we determine the initial values of the system. From Equation~\ref{eq: n equation of motion}, it is clear that we need both the initial position and velocity of a system to solve for the motion. In this case, the initial position of the cannonball can be considered to be the position of the cannon, i.e. \(r\h{0} = \h{0,h}^T\). The initial velocities are determined by some parameters, such that \(v\h{0} = \h{v_{x0},v_{y0}}\). Lastly, we need to determine the forces which we simply have to guess and tune until we are satisfied with the model. In this case, we introduce just the force of gravity.
	\begin{equation*}
		F_{\st[grav]}\h{t,r,\dot{r}} = F_{\st[grav]} = \mqty(0\\-mg).
	\end{equation*}
	Here, \(g\in\R\) is the acceleration due to gravity. As this is the only force we will be considering, we need to solve the following initial value problem:
	\begin{equation*}
		\dot{u} = \mqty(\dot{r}_x\\\dot{r}_y\\\dot{v}_x\\\dot{v}_y) = \mqty(v_x\\v_y\\0\\-g) = \mqty(0&0&1&0\\0&0&0&1\\0&0&0&0\\0&0&0&0)u + \mqty(0\\0\\0\\-g),\qquad u\h{0} = \mqty(0\\h\\v_{x0}\\v_{y0}).
	\end{equation*}
	Remark that this equation is of the form \(\dot{u} = Au + b\), thus we can find the solution by integrating, see Theorem 2.4.1 in \cite{MyintU1978}, such that
	\begin{equation*}
		u\h{t} = e^{tA}u\h{0} + \int_0^te^{\h{t - s}A}bds = \mqty(v_{x0}t\\h + v_{y0}t - \flatfrac{gt^2}{2}\\v_{x0}\\v_{y0} - gt).
	\end{equation*}
	Therefore, we get the motion for the cannonball with gravity
	\begin{equation}\label{eq: projectile motion gravity}
		r\h{t} = \mqty(v_{x0}t\\h + v_{y0}t - \flatfrac{gt^2}{2}).
	\end{equation}
	\begin{figure}
		\centering
		\includegraphics{img/Projectile_Gravity.pdf}
		\caption{The trajectory of a cannonball as predicted by Equation~\ref{eq: projectile motion gravity} and the calculated trajectory of Figure~\ref{fig: sketch projectile}. The same initial conditions were used in this figure. Remark that the trajectories are rather close to each other, but we see that the model predicts the range of the cannon to be further than the calculated range. Yet, our model is quite close to the numerical analysis.}
		\label{fig: projectile grav}
	\end{figure}
	We have plotted an example of such a motion in Figure~\ref{fig: projectile grav}. Here we see that our model is not quite perfect, but it does approximate the trajectory quite well.
\end{document}
	\end{example}
	\begin{example}\label{exp: atwood newton}
		\documentclass[class = report, crop = false]{standalone}
\usepackage{standalone}

\begin{document}
	\begin{figure}
		\centering
		\begin{subfigure}[t]{.49\textwidth}
			\centering
			\includegraphics{img/Atwood_Sketch.pdf}
			\caption{The configuration of a pulley system with two masses of different sizes hanging from it.}
			\label{fig: atwood sketch}
		\end{subfigure}
		\hfill
		\begin{subfigure}[t]{.49\textwidth}
			\centering
			\includegraphics{img/Atwood_ForceDiagram.pdf}
			\caption{The Force diagram associated with Figure~\ref{fig: atwood sketch}. The reference frame is drawn at the centre of the pulley.}
			\label{fig: atwood force diagram}
		\end{subfigure}
		\caption{Pulley system as described in Example~\ref{exp: atwood newton} with both a sketch and a force diagram.}
		\label{fig: atwood}
	\end{figure}
	Let us consider an Atwood machine, which is a system of two stationary masses, \(m_1\) and \(m_2\), hanging on a rope over a pulley, see Figure~\ref{fig: atwood sketch}. Assume that the rope and pulley are massless, there is no friction in the pulley, the rope does not slip on the pulley and the length of the rope is constant. Set the origin at the centre of the pulley such that the initial positions of the masses are given by \(r_1 = \h{-R,-y_{10}}^T\) and \(r_2 = \h{R,-y_{20}}^T\), where \(R\) is the radius of the pulley.
	
	The forces in the system are then the force of gravity acting on both the objects, such that \(F_{\footm{grav},\footm{m}_i} = \h{0,-m_ig}^T\), and a tension force \(T = \h{0,T}^T\) which is the same on both objects by the constraint on the length of the rope. These forces are drawn in Figure~\ref{fig: atwood force diagram}. We can translate this to the following equation of motion.
	\begin{equation*}
		\mqty(m_1a_{x1}\\m_1a_{y1}\\m_2a_{x2}\\m_2a_{y2}) = \mqty(0\\T + F_{\footm{grav},\footm{m}_1}\\0\\T + F_{\footm{grav},\footm{m}_2}) = \mqty(0\\T - m_1g\\0\\T - m_2g).
	\end{equation*}
	We notice here that we can ignore the \(x\)-coordinates as there is no net force acting in this direction. However, when solving this equation, we run into the problem that \(T\) is an unknown. Luckily, we can recover it by remarking that \(a_1 = -a_2\), which is a result of the fact that the length of the rope is constant. We can solve this equation for \(T\).
	\begin{align*}
		\dfrac{T - m_1g}{m_1} = a_1 &= - a_2= -\dfrac{T - m_2g}{m_2}\\
		 m_2T - m_1m_2g &= -m_1T + m_1m_2g\\
		 T &= \dfrac{2m_1m_2}{m_1 + m_2}g.
	\end{align*}
	Substituting this into the equation of motion, we get
	\begin{equation*}
		\mqty(a_{y1}\\a_{y2}) = \mqty(\flatfrac{2m_2g}{\h{m_1 + m_2}} - g\\\flatfrac{2m_1g}{\h{m_1 + m_2}} - g) = \mqty(\flatfrac{g\h{m_2 - m_1}}{\h{m_1 + m_2}}\\\flatfrac{g\h{m_1 - m_2}}{\h{m_1 + m_2}}).
	\end{equation*}
	Solving this system is then rather simple and results in the motion of the masses.
	\begin{equation}\label{eq: atwood result newton}
		r\h{t} = \mqty(r_1\h{t}\\r_2\h{t}) = \mqty(y_{10} + \flatfrac{\mu gt^2}{2}\\y_{20} - \flatfrac{\mu gt^2}{2}),\qquad \mu = \dfrac{m_2 - m_1}{m_1 + m_2}.
	\end{equation}
	Here, we can see that the mass difference is the main factor in the dynamics. If the difference is zero, the masses will stay stationary. If one of them is heavier, the greater mass will move downwards.
\end{document}
	\end{example}
\end{document}
\documentclass[class = article, crop = false]{standalone}
\usepackage{standalone}

\begin{document}
\section{Lagrangian Formalism}\label{sec: lagrangian}
In the last section, we discussed Newton's formulation of classical mechanics. Here, we also went over its application to two rather simple examples, Examples~\ref{exp: projectile}~and~\ref{exp: atwood newton}. We saw that it can be quite cumbersome to work with the geometry of vectors and constraints. Hence, people started developing scalar theories of classical mechanics. Lagrangian formalism is an example of such a scalar approach to classical mechanics. Where Newton focussed on force and momentum, Lagrange only need to account for the energies in a system. He can connect the motion of the objects with the energies in the system through Hamilton's principle, more accurately called the principle of stationary action. From this principle, we are then able to extract the Euler-Lagrange equations which form the equations of motion in this formalism. These Euler-Lagrange equations are even stronger as they let us use generalised coordinates which cover the space of configurations of the system. Before we go into the methodology of Lagrangian formalism, we will shortly discuss the scalar quantity of energy.

\subsection{Energy}
	Let us for now assume that we are still working on Euclidean space in Cartesian coordinates like in Newtonian mechanics. We distinguish two categories of energy: kinetic energy and potential energy. The first one is tied to the motion of an object while the second one results from the different interactions between objects and the system. We will introduce both of these objects through their interaction mechanism: work done by forces.
	\begin{definition}\label{def: work done by force}
		Let \(F\) be a force acting on an object which is moving along a curve \(\mathcal{C}\), we define the \textbf{work done by \(F\) on the object along \(\mathcal{C}\)} as
		\begin{equation*}
			W = \int_{\mathcal{C}}F\vdot ds = \int_{t_i}^{t_f}F\h{r\h{t},\dot{r}\h{t},t}\vdot\dot{r}\h{t}dt,
		\end{equation*}
		where \(r:\ha{t_i,t_f}\to\mathcal{C}\) is some parametrisation of the curve.
	\end{definition}
	Let us consider the work done by the net force acting on an object along the physical path it follows using Newton's equation of motion, we can then rewrite this integral
	\begin{equation*}
		W = \int_{t_i}^{t_f}F\h{r\h{t},\dot{r}\h{t},t}\vdot\dot{r}\h{t}dt = \int_{t_i}^{t_f}m\ddot{r}\h{t}\vdot\dot{r}\h{t}dt = \int_{t_i}^{t_f}d\h{\dfrac{1}{2}m\norm{\dot{r}\h{t}}^2}.
	\end{equation*}
	If we then define the quantity \(T\h{\dot{r}} = {m\norm{\dot{r}}^2}/{2}\), called the \textbf{kinetic energy}, we can express the work done on the object to be the change in kinetic energy of the object. If we add up the kinetic energy of all objects in a system, we get the \textbf{total kinetic energy} \(T_{\footm{tot}}\), which we often denote with just \(T\).
	
	Meanwhile, we could also try to integrate the work done by a single force. In very few cases is this of a neat form, hence, we will consider forces that lend themselves to an interpretation much like the kinetic energy.
	\begin{definition}
		A force \(F:\R[3]\times\R[3]\times\R\to\R[3]\) is called a \textbf{general conservative force} if there exists some \(U:\R[3]\times\R[3]\times\R\to\R\) such that given some path \(r:\R\to\R[3]\) we have
		\begin{equation}\label{eq: general potential}
			F_i\h{r\h{t},\dot{r}\h{t},t} = \dv{t}\h{\pdv{U}{\dot{r}_i}\h{r\h{t},\dot{r}\h{t},t}} - \pdv{U}{r_i}\h{r\h{t},\dot{r}\h{t},t}.
		\end{equation}
		Here, the subscript \(i\) denotes the \(i\)th component of the force, position and velocity vector. The function \(U\) is called the \textbf{general potential energy} of \(F\).
	\end{definition}
	\begin{example}\label{exp: potential of lorentz force}
		Take the Lorentz force acting on a particle with charge \(q\) moving along a path \(r:\R\to\R[3]\), i.e. the position is time-dependent, in an electric field \(E\) and magnetic field \(B\). The Lorentz force is then given by
		\begin{equation*}
			F_{\footm{Lorentz}}\h{r,\dot{r},t} = q\h{E\h{r,t} + \dot{r}\cross B\h{r,t}}.
		\end{equation*}
		Assume we have a scalar potential \(V\h{r,t}\) and vector potential \(A\h{r,t}\) that satisfy the following
		\begin{equation*}
			B\h{r,t} = \nabla\cross A\h{r,t},\qquad E\h{r,t} = -\nabla V\h{r,t} - \pdv{A}{t}\h{r,t}.
		\end{equation*}
		We can show that the potential of \(F\) in the sense of Equation~\ref{eq: general potential} is given by
		\begin{equation*}
			U\h{r,\dot{r},t} = q\h{V\h{r,t} - \dot{r}\cdot A\h{r,t}}.
		\end{equation*}
		We can check that this satisfies Equation \ref{eq: general potential} for the Lorentz force. The first component of the Lorentz force is given by
		\begin{equation*}
			F_{\footm{Lorentz},1} = q\h{-\pdv{V}{r_1} - \pdv{A_1}{t} + \dot{r}_2\h{\pdv{A_1}{r_2} - \pdv{A_2}{r_1}} - \dot{r}_3\h{\pdv{A_1}{r_3} - \pdv{A_3}{r_1}}}.
		\end{equation*}
		Meanwhile, the first component of the right-hand side of Equation \ref{eq: general potential} can be expressed as
		\begin{align*}
			\dv{t}\pdv{U}{\dot{r}_1} - \pdv{U}{r_1}
			&= q\h{-\dv{A_1}{t} - \pdv{V}{r_1} + \sum_i\dot{r}_i\pdv{A_i}{r_1}}\\
			&= q\h{-\pdv{A_1}{t} - \sum_i\dot{r}_i\pdv{A_1}{r_i} - \pdv{V}{r_1} + \sum_i\dot{r}_i\pdv{A_i}{r_1}}\\
			&= q\h{-\pdv{A_1}{t} - \pdv{V}{r_1} + \dot{r}_2\h{\pdv{A_2}{r_1} - \pdv{A_1}{r_2}} - \dot{r}_3\h{\pdv{A_1}{r_3} - \pdv{A_3}{r_1}}}.
		\end{align*}
		It follows from similar calculations that the other components are equal as well. Hence, \(U\) is indeed the general potential energy of the Lorentz force.
	\end{example}
	In the Lagrangian formalism, we do allow for generalised conservative forces. However, in the case that the potential does not explicitly depend on time, even the work simplifies as follows.
	\begin{align*}
		W
		&= \int_{t_i}^{t_f}F\h{r\h{t},\dot{r}\h{t},t}\vdot\dot{r}\h{t}dt\\
		&= \sum_{i = 1}^3\ha{\int_{t_i}^{t_f}\ha{\dv{t}\h{\pdv{U}{\dot{r}_i}\h{r\h{t},\dot{r}\h{t}}} - \pdv{U}{r_i}\h{r\h{t},\dot{r}\h{t}}}\cdot\dot{r}_i\h{t}dt}\\
		&= \sum_{i = 1}^3\Bigg[\int_{t_i}^{t_f}\Bigg[\dv{t}\h{\pdv{U}{\dot{r}_i}\h{r\h{t},\dot{r}\h{t}}\cdot\dot{r}_i\h{t}} - \pdv{U}{\dot{r}_i}\h{r\h{t},\dot{r}\h{t}}\cdot\ddot{r}_i\h{t}\\
		&\qquad - \pdv{U}{r_i}\h{r\h{t},\dot{r}\h{t}}\cdot\dot{r}_i\h{t}\Bigg]dt\Bigg]\\
		&= \sum_{i = 1}^3\Bigg[\eval{\pdv{U}{\dot{r}_i}\h{r\h{t},\dot{r}\h{t}}\cdot\dot{r}_i\h{t}}_{t_i}^{t_f} - \int_{t_i}^{t_f}\dv{t}\h{U\h{r\h{t},\dot{r}\h{t}}}dt\Bigg]\\
		&= \sum_{i = 1}^3\ha{\eval{\pdv{U}{\dot{r}_i}\h{r\h{t},\dot{r}\h{t}\cdot\dot{r}_i\h{t}}}_{t_i}^{t_f} - \eval{U\h{r\h{t},\dot{r}\h{t}}}_{t_i}^{t_f}}.
	\end{align*}
	Even though such a force does lead to a closed form for the work done by it, it can still be quite messy to work with. Therefore, we introduce the more simplistic \textbf{conservative force}, which is a force \(F:\R[3]\times\R\to\R[3]\) such there exists a \textbf{potential energy} \(U:\R[3]\times\R\to\R\) which satisfies \(F = -\nabla U\). In this case, the work simplifies to \(W = -\Delta U\).
	\begin{proposition}\label{prp: central force}
		A force, \(F:\R[3]\times\R\to\R[3]\), is called conservative if it can be written as
		\begin{equation}\label{eq: central force}
			F\h{r,t} = f\h{\norm{r},t}\dfrac{r}{\norm{r}},
		\end{equation}
		where \(f:\R\times\R\to\R\).
	\end{proposition}
	\begin{proof}
		Suppose that \(F\h{r,t}\) is as in Equation~\ref{eq: central force} and \(r_0\in\R[3]\). Define \(U:\R[3]\times\R\to\R\) as
		\begin{equation*}
			U\h{r,t} = \int_{\norm{r}}^\infty f\h{s,t}ds.
		\end{equation*}
		We assume that this integral exists, however, as a potential is defined up to a constant the upper limit can be chosen arbitrarily such that the integral does exist. We can deduce that
		\begin{equation*}
			\pdv{U}{r_i}\h{r,t} = \dv{r_i}\int_{\norm{r}}^\infty f\h{s,t}ds = -\pdv{\norm{r}}{r_i}f\h{\norm{r},t} = -f\h{\norm{r},t}\dfrac{r_i}{\norm{r}}.
		\end{equation*}
		Hence, we can deduce that \(U\) is indeed the potential of \(F\).
	\end{proof}		
	\begin{example}\label{exp: gravitational potential}
		Consider the force of gravity acting on an object placed at \(r_1\) and exerted by an object at \(r_2\), the force can be expressed as
		\begin{equation*}
			F\h{r_1} = \dfrac{Gm_1m_2}{\norm{r_1 - r_2}^2}\dfrac{r_2 - r_1}{\norm{r_1 - r_2}}.
		\end{equation*}
		We can define the potential energy \(U\h{r_1}\) as
		\begin{equation*}
			U\h{r_1} = \dfrac{Gm_1m_2}{\norm{r_2 - r_1}}.
		\end{equation*}
		Let us check that this is indeed the correct potential,
		\begin{equation*}
			-\nabla U\h{r_1} = \dfrac{Gm_1m_2}{\norm{r_1 - r_2}^2}\dfrac{r_2 - r_1}{\norm{r_1 - r_2}}.
		\end{equation*}
		Thus the force of gravity is conservative. See that it can indeed be written in terms of \(r = r_2 - r_1\), i.e. \(F\h{r} = \flatfrac{Gm_1m_2r}{\norm{r}^3}\). Furthermore, remark that the gravitational force on the second body exerted by the first can be obtained by taking the gradient with respect to \(r_2\), i.e. if we fix \(r_1\) and make \(r_2\) a variable, we obtain
		\begin{equation*}
			F\h{r_2} = -\nabla U\h{r_2} = \dfrac{Gm_1m_2}{\norm{r_1 - r_2}^2}\dfrac{r_1 - r_2}{\norm{r_1 - r_2}} = -F\h{r_1}.
		\end{equation*}
		This is exactly Newton's third law.
	\end{example}
	Example \ref{exp: gravitational potential} shows us that a potential is much more of a measure of the interaction rather than a quantity tied to a body. We would like to define the total potential energy, \(U_{\st[tot]}\) in such a way that \(\pdv*{U_{\st[tot]}}{r_i}\) is the net force on the \(i\)-th particle. Hence, the \textbf{total potential energy} of a system is given by \(U_{\st[tot]} = \sum_iU_i\), where the sum runs over all the interactions.
	
\subsection{Euler-Lagrange Equations}
	With the concepts of energy at hand, we can dive into Lagrangian formalism. This formalism lends itself to working with constrained systems in a more natural manner. In Newtonian mechanics, constraints led to imposing constraint forces, like the tension in Example~\ref{exp: atwood newton}. In Lagrangian formalism, we work around these constraints by choosing suitable coordinates which span all the possible configurations of the system. By choosing our coordinates wisely, we can often reduce the apparent dimensionality of the system. Such a system of coordinates is often denoted with \(q = \h{q_1,\ldots,q_n}\) instead of \(r = \h{r_1,\ldots,r_n}\). In such general coordinates, we can state the basic principle of Lagrangian formalism: Hamilton's principle.
	\begin{principle}
		The actual motion of a physical system, \(q:\ha{t_i,t_f}\to\R[3]\), is a stationary point of the \textbf{action integral} defined as
		\begin{equation*}
			S\h{q} = \int_{t_i}^{t_f}\lag\h{q\h{t},\dot{q}\h{t},t}dt.
		\end{equation*}
		Where \(\lag\h{q,\dot{q},t} = T\h{q,\dot{q},t} - U\h{q,\dot{q},t}\) is called the \textbf{Lagrangian} of the system.
	\end{principle}
	\begin{remark}
		Remark that in the previous section, the kinetic energy was only a function of \(\dot{r}\), but in general coordinates, we can have some dependence on the position. For example, if we use cylindrical coordinates, \(\h{x,y,z} = \h{r\sin\theta, r\sin\theta, z}\) one can deduce that \(T = \frac{1}{2}m\h{\dot{r}^2 + r^2\dot{\theta} + \dot{z}^2}\). Hence, it is dependent on both the velocities and the position. Furthermore, \(U\) can be considered in the sense of a general potential and can therefore be dependent on the velocity.
	\end{remark}
	\begin{remark}
		Remark that a Lagrangian only results in a well-posed mechanical situation if the stationary point of the action integral is uniquely for some boundary conditions. One can deduce that this implies that the Lagrangian must be a convex function in \(\dot{q}\), see \cite[p. 57]{Bolza1909} or \cite[Section 1.4]{Kielhoefer2018}.
	\end{remark}
	This principle is equivalent to the second law of motion posed by Newton. Finding the stationary points of an integral may seem like a convoluted way of finding the motions, and it is not even clear how this is equivalent to Newton's formalism. Luckily, both these problems are solved using the \textbf{Euler-Lagrange equations}.
	\begin{proposition}\label{prp: euler lagrange equations}
		For a Lagrangian \(\lag\), any stationary point \(q:\ha{t_i,t_f}\to\R\) of the action integral satisfies the following
		\begin{equation}\label{eq: el-equation one dimensional}
			\pdv{\lag}{q}=\dv{t}\pdv{\lag}{\dot{q}}.
		\end{equation}
		Moreover, any path that satisfies this condition is a stationary point.
	\end{proposition}
	\begin{proof}
		Suppose we are given a Lagrangian \(\lag = \lag\h{q,\dot{q},t}\). Let us define a stationary point of the action integral. First, we define the notion of a variation of a path \(q:\ha{t_i,t_f}\to\R\) as a path \(\eta:\ha{t_i,t_f}\to\R\) with \(\eta\h{t_i} = 0 = \eta\h{t_f}\). We can then vary the path \(q\) smoothly in the direction of \(\eta\) as \(\rho_{\alpha,\eta}:\ha{t_i,t_f}\to\R:t\mapsto q\h{t} + \alpha\eta\h{t}\). For each variation \(\eta\), we can express \(S\) as a function of \(\alpha\), 
		\begin{equation*}
			S_{\eta}\h{\alpha} = S\h{\rho_{\alpha,\eta}} = \int_{t_i}^{t_f}\lag\h{\rho_{\alpha,\eta}\h{t},\dot{\rho}_{\alpha,\eta}\h{t},t}dt.
		\end{equation*}
		We then call \(q\) a stationary point of \(S\) if for every variation \(\eta\) we have \(\dv*{S_{\eta}}{\alpha}|_{\alpha = 0} = 0\). Now suppose that \(q:\ha{t_i,t_f}\to\R\) is a stationary point of \(S\) and \(\eta\) is some variation. Using the Leibniz and product rule, we can deduce that
		\begin{align*}
			0
			&= \eval{\dv{S_\eta}{\alpha}}_{\alpha = 0} = \int_{t_i}^{t_f}\eval{\dv{\alpha}}_{\alpha = 0}\h{\lag\h{\rho_{\alpha,\eta}\h{t},\dot{\rho}_{\alpha,\eta}\h{t},\h{t}}}dt\\
			&= \int_{t_i}^{t_f}\h{\pdv{\lag}{q}\h{\rho_{0,\eta}\h{t},\dot{\rho}_{0,\eta},t}\eta\h{t} + \pdv{\lag}{\dot{q}}\h{\rho_{0,\eta}\h{t},\dot{\rho}_{0,\eta}\h{t},t}\dot{\eta}\h{t}}dt.
			\intertext{By rewriting \(\rho_{0,\eta}\h{t}\)  as \(q\h{t}\) and using partial integration on the second term in combination that \(n\h{t_i} = 0 = n\h{t_f}\) we get the following.}
			0 &= \int_{t_i}^{t_f}\eta\h{t}\h{\pdv{\lag}{q}\h{q\h{t},\dot{q}\h{t},t} - \dv{t}\pdv{\lag}{\dot{q}}\h{q\h{t},\dot{q}\h{t},t}}dt.
		\end{align*}
		As we have chosen our variation \(\eta\) arbitrarily, this is enough to conclude that
		\begin{equation*}
			\pdv{\lag}{q}\h{q\h{t},\dot{q}\h{t},t} - \dv{t}\pdv{\lag}{\dot{q}}\h{q\h{t},\dot{q}\h{t},t} = 0.
		\end{equation*}
		Hence, any path \(q\) that is a stationary point of the action integral satisfies Equation~\ref{eq: el-equation one dimensional}. Moreover, any path that satisfies this equation is a stationary point of the action integral by the same logic.
	\end{proof}
	If we generalise this to a higher dimensional system with some general coordinates \(q=\h{q_1,\ldots,q_n}\), we conclude that the physical path satisfies
	\begin{equation*}
		\pdv{\lag}{q_i}=\dv{t}\pdv{\lag}{\dot{q}_i},\ \forall 1\leq i\leq n.
	\end{equation*}
	See \cite[Proposition 1.4.1]{Kielhoefer2018} for a formal proof of this statement. The Euler-Lagrange equations show that given \(n\) generalised coordinates, we end up with \(n\) second-order differential equations we need to solve in order to recover the physical motions of the bodies.
	\begin{remark}
		In Cartesian coordinates, the Euler-Lagrange equations are equivalent to Newton's second law.
		\begin{align*}
			\pdv{\lag}{q} = \dv{t}\pdv{\lag}{\dot{q}}\Longleftrightarrow
			-\pdv{U}{q} = \dv{t}\h{m\dot{q} - \pdv{U}{\dot{q}}}\Longleftrightarrow
			\dot{p} = \dv{t}\pdv{U}{\dot{q}} - \pdv{U}{q} = F.
		\end{align*}
		Here we used that \(U\) is the general potential of the force acting on the body.
	\end{remark}
	Solving problems with this formalism is often seen as more straightforward and less error-prone. In practice we need to go through the following steps:
	\begin{enumerate}[label = {\alph*)}]
		\item Determine the kinetic and potential energies in an inertial frame.
		\item Determine the Lagrangian and translate it to some general coordinates for the system.
		\item Solve the Euler-Lagrange equation.
	\end{enumerate}
	We will now showcase the power of Lagrangian formalism using two examples.
	\begin{example}\label{exp: atwood lagrangian}
		\documentclass{standalone}
\usepackage{standalone}

\begin{document}
\begin{figure}
	\centering
	\includegraphics{img/Atwood_Coordinates.pdf}
	\caption{Generalised coordinates of an Atwood machine.}
	\label{fig: atwood coordinates}
\end{figure}
Let us consider the Atwood machine of Example~\ref{exp: atwood newton} again, i.e. we consider two stationary masses hanging from a rope which is suspended over a pulley. We assume that the rope and pulley are massless, the rope does not slip on the pulley, the pulley can rotate freely and the length of the rope is constant. We will now solve this using Lagrangian formalism. Define the coordinates \(x\) and \(y\) as in Figure~\ref{fig: atwood coordinates}. As the rope has a constant length, say \(L\), we get the relation \(x + y + R\pi = L\), implying that \(y = -x + \tilde{C}\). Hence, the system can be described using a single general coordinate \(x\). The kinetic energy of this system is given by
\begin{equation*}
	T\h{x} = \dfrac{1}{2}m_1\dot{x}^2 + \dfrac{1}{2}m_2\dot{y}^2 = \dfrac{1}{2}\h{m_1 + m_2}\dot{x}^2.
\end{equation*}
The potential energy is the sum of the gravitational potentials of both masses.
\begin{equation*}
	U = -m_1gx - m_2gy = -\h{m_1 - m_2}gx + C.
\end{equation*}
Here, we can set the constant to zero as a potential is determined up to a constant. This leads to the following Lagrangian.
\begin{equation}\label{eq: atwood lagrangian}
	\lag\h{x,\dot{x}} = \dfrac{1}{2}\h{m_1 + m_2}\dot{x}^2 + \h{m_1 - m_2}gx.
\end{equation}
If we enter this into the Euler-Lagrange equations we get the following differential equation:
\begin{equation*}
	\h{m_1 - m_2}g = \h{m_1 + m_2}\ddot{x}.
\end{equation*}
This is solvable for \(x\), which results in
\begin{equation}\label{eq: atwood result lagrangian}
	x\h{t} = \frac{1}{2}\dfrac{m_1 - m_2}{m_1 + m_2}gt^2 + x_0.
\end{equation}
This solves our system and we can see this is equivalent to the Newtonian case by comparing Equation~\ref{eq: atwood result lagrangian} to Equation~\ref{eq: atwood result newton}.
\end{document}
	\end{example}
	\begin{example}\label{exp: sliding block lagrangian}
		\documentclass[class = article, crop = false]{standalone}
\usepackage{standalone}
\begin{document}
	\begin{figure}
		\centering
		\includegraphics{img/Sliding_Slope.pdf}
		\caption{Sketch of a mass \(m_1\) on a wedge of mass \(m_2\) at an angle \(\theta\). Here, we allow the mass to slide along the wedge, and the wedge to slide horizontally. The origin is placed such that the right angle of the wedge coincides with it at \(t = 0\), and the associated \(x\) and \(y\)-axis are also given. The coordinates \(q_1\) and \(q_2\) are also drawn.}
		\label{fig: sliding block}
	\end{figure}
	Consider the case of a block of mass \(m_1\) sliding on a wedge of mass \(m_2\) that can move horizontally as sketched in Figure~\ref{fig: sliding block}, where we assume everything to be frictionless. If we were to solve this problem in Newtonian mechanics, we would have to deal with the awkward constraint force of the wedge acting on the mass and work out a lot of geometry. Luckily, we can circumvent this problem by using the energy methods of Lagrange and imposing the constraints through the coordinates we choose, as depicted with \(q_1\) and \(q_2\) in Figure~\ref{fig: sliding block}.
	
	We can set up the Lagrangian in the inertial frame, depicted by the \(x\) and \(y\) axes in Figure~\ref{fig: sliding block}. As these are Cartesian coordinates, this is quite simple.
	\begin{equation*}
		\lag\h{x_1,y_1,x_2,y_2,\dot{x}_1,\dot{y}_1,\dot{x}_2,\dot{y}_2} = \dfrac{1}{2}m_1\h{\dot{x}_1^2 + \dot{y}_1^2} + \dfrac{1}{2}m_2\h{\dot{x}_2^2 + \dot{y}_2^2} + m_1gy_1.
	\end{equation*}
	Next, we need to translate the Lagrangian to the general coordinates \(q_1\) and \(q_2\), this is given by the following, notice that these are defined up to some constant.
	\begin{alignat*}{3}
		&x_1 &&= -q_2 + q_1\cos\alpha,\qquad	&&y_1 = q_1\sin\alpha,\\
		&x_2 &&= -q_2,					&&y_2 = 0.
	\end{alignat*}
	With these coordinate transformations, we can translate our Lagrangian to the general coordinates, resulting in 
	\begin{equation*}
		\lag\h{q_1,q_2,\dot{q}_1,\dot{q}_2} = \dfrac{1}{2}\h{m_1 + m_2}\dot{q}_2^2 + \dfrac{1}{2}m_1\h{\dot{q}_1^2 - 2\dot{q}_1\dot{q}_2\cos\alpha} + m_1gq_1\sin\alpha.
	\end{equation*}
	The equations of motion are then given by the Euler-Lagrange equations,
	\begin{align}
		\label{eq: el equation wedge 1}m_1g\sin\alpha &= m_1\ddot{q}_1 - m_1\ddot{q}_2\cos\alpha,\\
		\label{eq: el equation wedge 2}0 &= \h{m_1 + m_2}\ddot{q}_2 - m_1\ddot{q}_1\cos\alpha.
	\end{align}
	We can express \(\ddot{q}_2\) as an equation of \(\ddot{q}_1\) using Equation~\ref{eq: el equation wedge 2} and entering this into Equation~\ref{eq: el equation wedge 1} we can recover a closed form for both \(\ddot{q}_1\) and \(\ddot{q}_2\)
	\begin{equation*}
		\ddot{q}_1 = \dfrac{g\sin\alpha}{1 - \dfrac{m_1\cos^2\alpha}{m_1 + m_2}}\quad\mbox{and}\quad\ddot{q}_2 = \dfrac{m_1g\sin\alpha\cos\alpha}{m_1\sin^2\alpha + m_2}.
	\end{equation*}
	Notice that both of these are constant, and one can thus easily solve these equations. We can check that these satisfy our intuition in the cases that \(m_2\to\infty\), \(m_2 = 0\), \(\alpha = 0\) or \(\alpha = \flatfrac{\pi}{2}\).
\end{document}
	\end{example}
	
\end{document}
\documentclass[class = article, crop = false]{standalone}
\usepackage{standalone}

\begin{document}
\section{Hamiltonian Formalism}\label{sec: hamiltonian}
	Up until now, we have developed methods to solve an \(n\)-body problem using at most \(3n\) differential equations, either Newton's second law or the Euler-Lagrange equations. In the Lagrangian formalism, we could lower the dimensionality of the problem by choosing generalised coordinates, which use the symmetries of our problem. However, both Newtonian and Lagrangian formalism end up giving us second-order differential equations, which are not very insightful. Hence, we would like to reduce the order of our system naturally. We will show that we can do this by introducing the general momenta of a Lagrangian system as the new coordinates, which results in the Hamiltonian through the Legendre transform. Before we go this route, we will introduce the Hamiltonian in Cartesian coordinates.
	
	\subsection{Cartesian Hamiltonian Mechanics}
		Let us consider an \(n\)-body system for which we have the total energy functional given by a function \(\ham\h{r_1,\ldots,r_n,p_1,\ldots,p_n,t}\), where \(r_i\) is the position of the \(i\)th body and \(p_i\) the momentum of the \(i\)th body. We can write this in terms of the kinetic en potential energy, where we assume that the kinetic energy is only a function of \(p_i\) and the potential energy a function of \(r_i\) and \(t\)
		\begin{equation*}
			\ham\h{r_1,\ldots,r_n,p_1,\ldots,p_n,t} = T\h{p_1,\ldots,p_n} + U\h{r_1,\ldots,r_n,t} = \sum_{i}\dfrac{p_i^2}{2m_i} + U\h{r_1,\ldots,r_n,t}.
		\end{equation*}
		Remark that the potential is chosen such that \(F_i = \pdv*{U}{r_i}\), it then follows from the definition of momentum and Newton's second law that
		\begin{equation}\label{eq: hamiltons equations on rn}
			\pdv{\ham}{p_i} = \dfrac{p_i}{m_i} = \dot{r}_i,\quad \pdv{\ham}{r_i} = \pdv{U}{r_i} = -F_i = -\dot{p}_i.
		\end{equation}
		This gives us \(6n\) first-order differential equations to solve to obtain the motion of the \(n\)-bodies described by the energy functional. Furthermore, using these relations we obtain
		\begin{equation*}
			\dv{\ham}{t} = \dot{r}_i\pdv{\ham}{r_i} + \dot{p}_i\pdv{\ham}{p_i} + \pdv{\ham}{t} = \pdv{\ham}{t}.
		\end{equation*}
		Hence, the Hamiltonian is conserved as long as it does not depend on time directly.
		
	\subsection{The Hamiltonian in Generalised Coordinates}
		We will now try to extend the discussion of the previous section to work with general coordinates. Remark that the energy functional in the previous section was dependent on the position, momentum and time. We would like to replicate this in general coordinates to obtain a similar equation to Equation~\ref{eq: hamiltons equations on rn}. To do this, we will first have to determine what our quantity of momentum is in generalised coordinates. We will define this in relation to a Lagrangian. Given a Lagrangian \(\lag\) in some generalised coordinates \(\h{q_1,\ldots,q_n}\), we define the generalised momentum associated with a coordinate \(q_i\) as
		\begin{equation*}
			p_i = \pdv{\lag}{\dot{q}_i}.
		\end{equation*}
		This definition might seem odd, but it works correctly in Cartesian coordinates.
		\begin{example}
			Consider the Lagrangian of \(n\) non-interacting objects. In this case, the Lagrangian in Cartesian coordinates is given by
			\begin{equation*}
				\lag\h{r_1,\ldots,r_n,\dot{r}_1,\ldots,\dot{r}_n,t} = \dfrac{1}{2}\sum_im_i\norm{\dot{r}_i}^2.
			\end{equation*}
			Hence, the generalised momentum associates with \(r_i\) is simply the momentum of the object
			\begin{equation*}
				p_i = \pdv{\lag}{\dot{r}_i} = m_i\dot{r}_i.
			\end{equation*}
			The definition of the generalised momenta coincides with the usual definition when working in Cartesian coordinates when working with non-interacting particles.
		\end{example}
		We would then like to naturally transform the Lagrangian \(\lag\h{q_1,\ldots,q_n,\dot{q}_1,\ldots,\dot{q}_n,t}\) to a function \(\ham\h{q_1,\ldots,q_n,\pdv{\lag}{\dot{q}_1},\ldots,\pdv{\lag}{\dot{q}_n},t} = \ham\h{q_1,\ldots,q_n,p_1,\ldots,p_n,t}\), such that there is a change of dependent variable. This transformation can be formalised using the Legendre transform. 
		\subsubsection{Legendre Transform}
			Consider a function \(f:V\to\R\), most often we have \(V = \R[n]\), we define the \textbf{Legendre transform} of \(f\) as the function \(\dual{f}:S\subset\dual{V}\to\R\) for which
			\begin{equation*}
				\dual{f}\h{\alpha} = \sup_{v\in V}\h{\alpha\h{v} - f\h{v}}.
			\end{equation*}
 			Remark that this function is defined for all \(\alpha\in\dual{V}\) for which the supremum is finite.
 			\begin{example}
 				Consider the function \(f:\R\to\R:x\mapsto e^x\), we can recognise that \(\dual{R}\cong\R\) such that we can define the Legendre transform as
 				\begin{equation*}
 					\dual{f}\h{\dual{x}} = \sup_{x\in\R}\h{\dual{x}x - e^x},
 				\end{equation*}
 				where \(\dual{x}\in \dual{I}\) which is some domain that we still have to determine. Let us first figure out what \(\dual{f}\) is by calculating the supremum using the derivative with respect to \(x\) and setting it equal to zero.
 				\begin{equation*}
 					0 = \dv{x}\h{\dual{x}x - e^x} = \dual{x} - e^x.
 				\end{equation*}
 				Hence we see that \(\dual{x} = e^x\) gives a critical point, and as \(-e^x < 0\) for all \(x\) it follows that we achieve a maximum at \(x = \ln\h{\dual{x}}\). Thus we retrieve our Legendre transform
 				\begin{equation*}
 					\dual{f}\h{\dual{x}} = \h{\dual{x} - 1}\ln\h{\dual{x}}.
 				\end{equation*}
 				Remark that this function only exists for \(\dual{x} \in \h{0,\infty} = \dual{I}\). 
 			\end{example}
 			\begin{example}\label{exp: legendre}
 				Take a function \(g:\R[2]\to\R:\h{x,y}\mapsto g\h{x,y}\) which is convex in \(y\) and for a \(x\in \R\) defined \(f_x:\R\to\R:y\mapsto g\h{x,y}\). Remark that \(\dv*{f_x}{y}\h{y} = \pdv*{g}{y}\h{x,y}\). We can then determine the Legendre transform of \(f_x\), which was defined as
 				\begin{equation*}
 					\dual{f_x}\h{\dual{y}} = \sup_{y\in\R}\h{\dual{y}y - f_x\h{y}}.
 				\end{equation*}
 				We can again determine this supremum by differentiating with respect to \(y\). It then follows that
 				\begin{equation*}
 					0 = \dv{y}\h{\dual{y}y - f_x\h{y}} = \dual{y} - \dv{f_x}{y}\h{y}\implies \dual{y} = \dv{f_x}{y}\h{y}.
 				\end{equation*}
 				As \(g\) is convex in \(y\), it follows that \(y = \h{\dv*{f_x}{y}}^{-1}\h{\dual{y}}\) is the maximum. We can now transform \(g\h{x,y}\) to a function \(h\h{x,\pdv*{g}{y}}\) by defining
 				\begin{equation*}
 					h\h{x,\pdv{g}{y}} = h\h{x,\dual{y}}= \dual{f_x}\h{\dual{y}} = \dual{y}\h{\dv{f_x}{y}}^{-1}\h{\dual{y}} - g\h{x,\h{\dv{f_x}{y}}^{-1}\h{\dual{y}}}.
 				\end{equation*}
 				This leads to a natural transformation from \(g\h{x,y}\) to \(h\h{x,\pdv*{g}{y}}\). Furthermore, the differentials of \(g\) and \(h\) are given as
 				\begin{equation*}
 					dg = udx + vdy\implies dh = xdu - vdy.
 				\end{equation*}
 				Remark that this is not merely a coordinate transformation, but also a transformation of the space on which the functions act.
 			\end{example}
 			In the section on Lagrangian formalism, we remarked that a Lagrangian had to be convex in \(\dot{q}\) if it were to result in a well-posed problem. Hence, we can use Example~\ref{exp: legendre} to transform the Lagrangian in the manner discussed before. We define the \textbf{Hamiltonian} of a Lagrangian \(\lag\) as
			\begin{equation}\label{eq: hamiltonian from lagrangian}
				\ham\h{q_1,\ldots,q_n,p_1,\ldots,p_n,t} = \sum_{i = 1}^np_i\dot{q}_i - \lag\h{q_1,\ldots,q_n,\dot{q}_1,\ldots,\dot{q}_n,t},\ p_i = \pdv{\lag}{\dot{q}_i}.
			\end{equation}
			Hence, the Hamiltonian can be seen as the Legendre transform of the Lagrangian. We can now determine the differential along a physical motion in two manners: directly and using equation \ref{eq: hamiltonian from lagrangian}. If we calculate it directly, we find
			\begin{equation}\label{eq: differential hamiltonian 1}
				d\ham = \sum_{i = 1}^n\pdv{\ham}{q_i} + \sum_{i = 1}^n\pdv{\ham}{p_i}dp_i + \pdv{\ham}{t}dt.
			\end{equation}
			However, if we calculate it using Equation~\ref{eq: hamiltonian from lagrangian}, we find that
			\begin{equation*}
				d\ham
				= \sum_{i = 1}^np_id\dot{q}_i + \sum_{i = 1}^n\dot{q}_idp_i - d\lag.
			\end{equation*}
			We can further expand this expression by first determining the differential of the Lagrangian.
			\begin{equation*}
				d\lag = \sum_{i = 1}^n\pdv{\lag}{q_i}dq_i + \sum_{i = 1}^n\pdv{\lag}{\dot{q}_i}d\dot{q}_i + \pdv{\lag}{t}dt.
			\end{equation*}
			Combining these two equations results in the differential of the Hamiltonian:
			\begin{align}
				d\ham
		\notag	&= \sum_{i = 1}^np_id\dot{q}_i + \sum_{i = 1}^n\dot{q}_idp_i - \sum_{i = 1}^n\pdv{\lag}{q_i}dq_i - \sum_{i = 1}^n\pdv{\lag}{\dot{q}_i}d\dot{q}_i - \pdv{\lag}{t}dt.
				\intertext{Again remark that \(p_i = \pdv*{\lag}{\dot{q}_i}\) and that it follows from the Euler-Lagrange equations that the physical path the motion follows satisfies \(\dot{p}_i = \dv*{t}\h{\pdv*{\lag}{\dot{q}_i}} = \pdv*{\lag}{q_i}\).}
				d\ham
		\notag	&= \sum_{i = 1}^np_id\dot{q}_i + \sum_{i = 1}^n\dot{q}_idp_i - \sum_{i = 1}^n\dot{p}_idq_i - \sum_{i = 1}^np_id\dot{q}_i - \pdv{\lag}{t}dt\\
\label{eq: differential hamiltonian 2}
				&= \sum_{i = 1}^n\dot{q}_idp_i - \sum_{i = 1}^n\dot{p}_idq_i - \pdv{\lag}{t}dt.
			\end{align}
			Comparing Equation~\ref{eq: differential hamiltonian 1} and~\ref{eq: differential hamiltonian 2} gives us the \(2n + 1\) equations called \textbf{Hamilton's equations}:
			\begin{equation*}
					-\pdv{\lag}{t}	= \pdv{\ham}{t},\quad 
					-\dot{p}_i		= \pdv{\ham}{q_i},\quad
					\dot{q}_i 		= \pdv{\ham}{p_i},\qquad\forall 1\leq i\leq n.
			\end{equation*}
			Solving a mechanical problem comes down to solving this system of equations, most importantly the last \(2n\), to obtain the motion of the objects. In practice solving a problem goes as follows:
			\begin{enumerate}[label = {\alph*)}]
				\item Determine the kinetic and potential energies in an inertial frame.
				\item Determine the Lagrangian and translate it to some general coordinates for the system.
				\item Derive the generalised momenta from the Lagrangian and solve for the \(\dot{q}\)'s as functions of \(p\)'s and \(q\)'s.
				\item Determine the Hamiltonian using Equation~\ref{eq: hamiltonian from lagrangian}.
				\item Solve Hamilton's equations.
			\end{enumerate}
			Let us showcase this method using an example.
			\begin{example}
				\documentclass{standalone}
\usepackage{standalone}

\begin{document}
	Let us again consider the Atwood machine of Examples~\ref{exp: atwood newton}~and~\ref{exp: atwood lagrangian}. We already showed how to derive the Lagrangian in the coordinate described in Figure~\ref{fig: atwood coordinates}, see Equation~\ref{eq: atwood lagrangian}. To recover the Hamiltonian, we'll have to determine the generalised momentum
	\begin{equation*}
		p = \pdv{\lag}{\dot{x}} = \h{m_1 + m_2}\dot{x}.
	\end{equation*}
	Using the Legendre transform, we then obtain the Hamiltonian
	\begin{equation*}
		\ham\h{x,p,t} = p\dot{x} - \lag\h{x,\dot{x},t} = \dfrac{p^2}{2\h{m_1 + m_2}} - \dfrac{m_1 - m_2}gx.
	\end{equation*}
	This results in the following equations for \(\dot{x}\) and \(\dot{p}\)
	\begin{equation*}
		\dot{x} = \pdv{\ham}{p} = \dfrac{p}{m_1 + m_2},\quad \dot{p} = -\pdv{\ham}{x} = \h{m_1 - m_2}g.
	\end{equation*}
	We can see that these are equivalent to Equations~\ref{eq: atwood result newton} and~\ref{eq: atwood result lagrangian}.
\end{document}
			\end{example}
\end{document}
\end{document}