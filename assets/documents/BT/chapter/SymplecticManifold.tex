\documentclass[class = report, crop = false]{standalone}
\usepackage{standalone}

\begin{document}
	
\chapter{Symplectic Geometry on Manifolds}\label{chp: symplectic manifolds}
In the previous chapter, we introduced linear symplectic geometry in terms of symplectic vector spaces. Now we will evolve this to a more general geometric theory by extending it to manifolds. This is done naturally by using the locally linear structure of the tangent space, hence, imposing a symplectic form as a non-degenerate \(2\)-form which we will also require to be closed. We will see how this theory differs from its linear counterpart in the main focus of this chapter: Darboux's theorem. This is the non-linear version of Theorem~\ref{thm: linear darboux}, but it requires a much more involved theory on differential forms. In this chapter, we follow the structure and definitions of chapters 1, 7 and 8 in \cite{CannasdaSilva2008}. Furthermore, we will use differential forms and the exterior derivative in this chapter, the definitions of which are recalled in Section \ref{sec: exterior derivative}. If one is not yet familiar with these concepts, one should refer to more detailed sources like \cites{Lee2013, Marcut2017, Tu2011}.

\section{Basics and Definitions}\label{sec: basics symplectic manifolds}
	We will start this section with the definition of a symplectic manifold, followed by some examples. As mentioned, we want to generalise a linear algebra structure to differential geometry. Hence, we want to make use of the linear structure a manifold has in the form of its tangent space. This method is comparable to that of Riemannian geometry, which generalises an inner product space using a symmetric covariant \(2\)-tensor field that is positive definite everywhere, such that the tensor field is an inner product at each point. Similarly, we generalise the linear symplectic forms by looking at a tensor that is a linear symplectic form at each point.
	\begin{definition}\label{def: symplectic manifold}
		Let \(\m\) be a manifold, a \(2\)-form \(\omega\) is \textbf{symplectic} if it is closed, i.e. \(d\omega = 0\) and \(\omega_p:T_p\m\times T_p\m\to\R\) is symplectic for all \(p\in\m\) in the sense of vector spaces. The pair \(\h{\m,\omega}\) is then called a \textbf{symplectic manifold}.
	\end{definition}
	Once again we will go over some basic examples of symplectic vector spaces before we proceed. The first two of these are the equivalents to Examples~\ref{exp: trivial symplectic vector space} and~\ref{exp: linear canonical symplectic form}.
	\begin{example}\label{exp: trivial manifold}
		Let us define the \textbf{trivial symplectic manifold of dimension \(2n\)}. Take the manifold \(\R[2n]\) and take the global coordinates \(\symcor{x}{y}\). Define the \(2\)-form \(\omega_{\st,2n}\) as
		\begin{equation*}
			\omega_{\st,2n} = \sum_{i = 1}^ndx^i\wedge dy^i.
		\end{equation*}
		This form is symplectic as for each \(p\in\m\) the set
		\begin{equation*}
			\hv{\eval{\pdv{x^1}}_p,\cdots,\eval{\pdv{x^n}}_p,\eval{\pdv{y^1}}_p,\cdots,\eval{\pdv{y^n}}_p},
		\end{equation*}
		forms a symplectic basis for \(T_p\m\). Furthermore, one can easily check that this form is closed with a calculation.
		\begin{equation*}
			d\omega_{\st,2n} = d\sum_{i = 1}^ndx^i\wedge dy^i = \sum_{i = 1}^nd\h{dx^i\wedge dy^i} = \sum_{i = 1}^n\h{d^2}x^i\wedge dy^i - dx^i\wedge \h{d^2}y^i = 0.
		\end{equation*}
		This proves that \(\h{\R[2n],\omega_{\st,2n}}\) is a symplectic manifold. We call \(\omega_{\st,2n}\) a \textbf{trivial symplectic form} of which we sometimes mention the dimension.
	\end{example}
	\begin{example}\label{exp: canonical manifold}
		Let \(\m\) be an \(n\)-dimensional manifold and let \(\cotang\) be its \textbf{cotangent bundle}, which is defined as
		\begin{equation*}
			\cotang = \coprod_{p\in\m}\dual{\h{\loctang{p}}} = \coprod_{p\in\m}\loccotang{p}.
		\end{equation*}
		Firstly, we define a \textbf{tautological \(1\)-form on \(\cotang\)}, denoted by \(\alpha_{\taut}\), pointwise as
		\begin{equation*}
			{\alpha_{\taut}}_{\h{p,\xi}} = \pull[\h{p,\xi}]{d\pi}\h{\xi},
		\end{equation*}
		We then define the \textbf{canonical symplectic form on \(\cotang\)}, denoted as \(\omega_{\can}\), as
		\begin{equation*}
			\omega_{\can} = -d\alpha_{\taut}.
		\end{equation*}
		It should be clear that this is a closed and alternating \(2\)-tensor. To show that it is non-degenerate we will consider it in coordinates. Let us first look at the structure of \(\cotang\), which has some natural coordinates induced by a chart on \(\m\). Suppose that \(\h{U,\h{x^1,\ldots,x^n}}\) is a chart around \(p\in\m\). Via the differentials, \(\h{dx^1_p,\ldots,dx^n_p}\), we induce the following coordinates on \(\cotang[\m[U]]\):
		\begin{equation*}
			\cotang[\m[U]]\to\R[2n]:\xi_idx^i|_p\mapsto\h{x^1\h{p},\ldots,x^n\h{p},\xi_1,\ldots,\xi_n}.
 		\end{equation*}
	 	In these coordinates the projection is trivial and the differential is of the form \(\pull[\h{x,\xi}]{d\pi}\h{dx^i} = dx^i\). Hence, in coordinates the tautological \(1\)-form is given by
		\begin{equation*}
			{\alpha_{\taut}}_{\h{x,\xi}} = \xi_i dx^i.
		\end{equation*}
		As the charts are smooth, the tautological form is also smooth. Furthermore, the coordinate form of \(\alpha_{\taut}\) implies that \(\omega_{\can}\) is given by
		\begin{equation*}
			\omega_{\can} = \sum_{i = 1}^ndx^i\wedge d\xi_i,
		\end{equation*}
		in the coordinates \(\h{x^1,\ldots,x^n,\xi_1,\ldots,\xi_n}\). This form is locally equivalent to that of Example~\ref{exp: trivial manifold}, therefore it is symplectic by the same reasoning. It follows that \(\h{\cotang, \omega_{\can}}\) is a symplectic manifold.
	\end{example}
	\begin{proposition}\label{prp: existence canonical form}
		Every manifold admits a symplectic form on its cotangent bundle.
	\end{proposition}
	\begin{proof}
		This follows directly from the construction of the canonical symplectic form in Example~\ref{exp: canonical manifold}.
	\end{proof}
	There are of course many different symplectic manifolds and even different symplectic forms on a single manifold.
	\begin{example}\label{exp: 2-sphere}
		Take the \(2\)-sphere, denoted by \(S^2\). For the construction of the symplectic form, we identify \(S^2\) as a subset of \(\R[3]\), which we endow with the dot product, denoted by \(\inp{\cdot}{\cdot}\), and the cross product, denoted by \(\cross\). We then define the \(\mathring{\omega}\) at a point \(p\in S^2\) as
		\begin{equation*}
			 \mathring{\omega}_p:\loctang[S^2]{p}\times\loctang[S^2]{p}\to\R:\h{u,v}\mapsto\inp{p}{u\cross v}.
		\end{equation*}
		By the skew-symmetry and bilinearity of the cross product and the linearity of the inner product, it follows that \(\mathring{\omega}_p\) is a symplectic form on \(\loctang[S^2]{p}\). As the inner product and cross product are linear in both components, it is also smooth. Due to it being top-degree, it is closed. It follows that \(\mathring{\omega}\) is a symplectic form and \(\h{S^2,\mathring{\omega}}\) is a symplectic manifold. 
	\end{example}
	\begin{example}\label{exp: exotic}
		Let us define a symplectic form on \(\R[2]\) that is not the trivial symplectic form. Take the global coordinates \(\h{x,y}\) of \(\R[2]\) and define the \(2\)-form \(\omega_{\st[exp]}\) as
		\begin{equation*}
			\omega_{\st[exp]} = e^{-\h{x^2 + y^2}}dx\wedge dy.
		\end{equation*}
		We can quite easily recognise that this is a skew-symmetric non-degenerate form. Furthermore, as it is of top-degree it is also closed. Therefore, \(\h{\R[2],\omega_{\st[exp]}}\) is a symplectic vector space.
	\end{example}
	With these examples, we will again go on to the geometric side of symplectic manifolds.

\section{Symplectomorphisms}\label{sec: symplectomorphism}
	With this new category of structured objects, we look for a class of functions preserving our structure, which we will call symplectomorphisms. As they should be functions preserving manifolds they ought to be diffeomorphisms. Furthermore, they should preserve the structure added to the tangent spaces as well. Hence, we want the function to satisfy the commutative diagram of Figure~\ref{comdia: symplectomorphism}. This is done using the pullback of differential forms, see \cite[p. 360]{Lee2013}. This then leads to the following definition.
	\begin{figure}
		\centering
		\includegraphics{img/CommutativeDiagram_Symplectomorphism.pdf}
		\caption{A commutative diagram showing the way the symplectomorphism
			\(F:\m\to\m[N]\) should commute with the symplectic forms. Here, \(dF_p\) denotes
			the pointwise pushforward of \(F\).}
		\label{comdia: symplectomorphism}
	\end{figure}
	\begin{definition}\label{def: symplectomorphism}
		Let \(\h{\m_1,\omega_1}\) and \(\h{\m_2,\omega_2}\) be symplectic manifolds. A \textbf{symplectomorphism} between \(\h{\m_1,\omega_1}\) and \(\h{\m_2,\omega_2}\) is a diffeomorphism \(\phi:\m_1\to\m_2\) such that \(\pull{\phi}\omega_2 = \omega_1\). If such a symplectomorphism exists, then \(\h{\m_1,\omega_1}\) and \(\h{\m_2,\omega_2}\) are \textbf{symplectomorphic}.
	\end{definition}
	For vector spaces, the next important result was Theorem~\ref{thm: existence symplectic basis}, which described how we can find a basis in which the symplectic form is written as
	\begin{equation*}
		\ha{\Omega}_\beta = \mqty(0&\id\\-\id&0).
	\end{equation*}
	Implying that any symplectic vector space is isomorphic to \(\h{\R[2n],\Omega_{\st,2n}}\) for some \(n\). However, this runs into a problem when we try it with manifolds. For example, consider \(\h{\R[2],\omega_{\st,2}}\) and \(\h{S^2,\mathring{\omega}}\) as in Example~\ref{exp: 2-sphere}. If we could construct a symplectomorphism between these spaces, it would in particular be a homeomorphism between \(S^2\) and \(\R[2]\). However, as \(S^2\) is bounded and closed in \(\R[3]\) it is compact and homeomorphisms preserve compactness. This would then imply that \(\R[2]\) is compact, which leads to a contradiction. Thus, a symplectomorphism between \(\h{\R[2],\omega_{\st,2}}\) and \(\h{S^2,\mathring{\omega}}\) can not exist.
	
	We might then expect that we should at least be able to find a symplectomorphism between symplectic forms on the same manifold, but Example~\ref{exp: exotic} showcases that this is also not true. If a symplectomorphism \(\phi\) between \(\h{\R[2],\omega_{\st,2}}\) and \(\h{\R[2],\omega_{\st[exp]}}\) as in Example~\ref{exp: exotic}, would exist, it should preserve integrals, see \cite[Proposition 16.6]{Lee2013}. Hence the following equation should hold.
	\begin{equation}\label{eq: equivalence under symp}
		\int_{\R[2]}\omega_{\st[exp]} = \int_{\R[2]}\pull{\phi}\omega_{\st[exp]} = \int_{\R[2]}\omega_{\st,2}.
	\end{equation}
	Now we can calculate both sides of the equation. The integral of \(\omega_{\st,2}\) diverges as \(\omega_{\st,2}\) is a positive constant on the whole of \(\R[2]\). Meanwhile, the integral of \(\omega_{\st[exp]}\) can be calculated using a simple change of coordinates and a substitution as in the proof of the Gauss integral.
	\begin{equation*}
		\int_{\R[2]}\omega_{\st[exp]} = \int_{-\infty}^\infty\int_{-\infty}^\infty e^{-\h{x^2 + y^2}}dxdy = \int_{0}^{2\pi}\int_{0}^{\infty}e^{-r^2}rdrd\phi = \pi\int_{0}^\infty e^{-u}du = \pi.
	\end{equation*}
	As this integral does not diverge, we can conclude from Equation \ref{eq: equivalence under symp} that a symplectomorphism between \(\h{\R[2],\omega_{\st,2}}\) and \(\h{\R[2],\omega_{\st[exp]}}\) can not exist. Furthermore, this implies that these two symplectic manifolds are not symplectomorphic.
	
	From these examples, we can conclude that we can not simply study the global behaviour of \(\h{\R[2],\omega_{\st,2}}\) in symplectic geometry. The construction of a symplectomorphism is not even limited to the topological properties of the manifolds and may even fail for two symplectic forms on the same manifold.
	
\section{Normal Coordinates}\label{sec: normal coordinates}
	We saw in the previous section that there are many symplectic manifolds which are not symplectomorphic to \(\h{\R[2n],\omega_{\st,2n}}\).	This means we cannot encapsulate the global behaviour of symplectic manifolds by just studying this simple model space. Luckily, the structure of a manifold lends itself perfectly to studying local behaviour instead. To this extent, we will look for certain coordinates in which the symplectic form is of a `nice' form. Such coordinates are generally called normal coordinates.
	
	To get an idea of what is possible, we will once again compare symplectic manifolds to Riemannian manifolds. We already remarked that they are constructed similarly; take a structure on a vector space and implement it at each point to a tangent space. For the symplectic manifold, we use symplectic forms and for the Riemannian case, we consider inner products. For both linear structures, there are no invariants that determine the structures, i.e. the inner product is diagonal and the linear symplectic form is anti-diagonal and skew-symmetric, see Theorem~\ref{thm: existence symplectic basis}. We would hope that this simplistic structure would naturally be inherited by the manifold structure locally. Riemannian manifolds already show that this is not necessarily true. At a point, we can find a chart in which the metric becomes diagonal, but when extended to a neighbourhood this fails. Instead of a diagonal form, it becomes a function of the Riemannian curvature. We will not prove the following theorem, but one can refer to \cite[Corollary 9.8]{Gray2004} for a thorough exposition and proof.
	\begin{theorem}\label{thm: local riemannian metric}
		Let \(\h{\m,g}\) denote an Riemannian manifold with curvature tensor \(R_{ijkl}\). Around any point \(p\in\m\) there is a coordinate chart \(\h{U,\h{x^i}}\) centred at \(p\), such that the metric is given by
		\begin{equation*}
			g_{ij}|_U = \delta_{ij} - \dfrac{1}{3}\sum_{kl = 1}^nR_{kilj}\h{p}x^kx^l - \dfrac{1}{6}\sum_{klm = 1}^n\nabla_kR_{limj}\h{p}x^kx^lx^m + h\circ x,
		\end{equation*}
		where \(\lim_{x\to 0}\flatfrac{h\h{x}}{\norm{x}^5} = 0\).
	\end{theorem}
	This theorem tells us that a Riemannian manifold has some local invariant, namely the curvature, that determines whether two Riemannian manifolds are locally isometric. We will now prove this theorem as it is not in our current domain of study. However, we can give an informal argument which explains why we expect to find such an invariant. This is called the counting argument and was already given by Riemann in \cite{Riemann1921}, see \cite[Chapter 4]{Spivak1979} for a translation. In this `proof', he tries to count the number of degrees of freedom in the metric and the amount we can determine under a transformation of coordinates.
	
	Suppose that \(\h{\m,g}\) is an \(n\) dimensional Riemannian manifold. Around a point \(p\in\m\) we can write \(g\) is some coordinate chart \(\h{U,\h{x^i}}\) as \(g|_U = \sum_{i = 1}^ng_{ij}dx^idx^j\), such that \(g\) is fully determined in the neighbourhood \(U\) by \(n^2\) functions. This implies that we have \(n^2\) degrees of freedom, but we lose \(\flatfrac{n\h{n - 1}}{2}\) degrees of freedom to the symmetry of the metric. Hence, there are just \(n^2 - \flatfrac{n\h{n - 1}}{2} = \flatfrac{n\h{n + 1}}{2}\) degrees of freedom left. If we then take into account the ambiguity of choosing coordinates, we lose another \(n\) degrees of freedom. This leaves us with just \(\flatfrac{n\h{n - 1}}{2}\) degrees of freedom we can not determine further. Hence, we have shown that there are some degrees of freedom in our choice of metric, which we conveniently capture in the Riemannian curvature.
	
	This goes to show that we can not ensure the existence of a chart in which the metric is given by \(g|_U = \sum_{i = 1}^ndx^idx^i\). A Riemannian manifold does therefore not directly inherit the properties of the inner product locally. If there exists a map from the Riemannian manifold to the Euclidean space which locally restricts to an isometry, we call this Riemannian manifold flat. It should be clear that there exist some coordinates \(\h{U,\h{x^i}}\) on a flat Riemannian manifold such that the metric is given by \(g|_U = \sum_{i = 1}^ndx^idx^i\). We can then find a direct connection between the flatness of a Riemannian manifold and its Riemannian curvature tensor, something we will again not prove here, but a proof can be found in \cite[Theorem 7.10]{Lee2019}. 
	\begin{proposition}
		A Riemannian manifold is flat if and only if its Riemannian curvature tensor vanishes identically.
	\end{proposition}
	This is a rather simple condition for a Riemannian manifold to be flat. However, we will show that this holds in general for symplectic forms in a theorem that is called Darboux's theorem. A proof of this was first given by Jean Gaston Darboux in 1882 \cite{Darboux1882}, but later it was simplified by Weinstein \cite{Weinstein1971} based on a method developed by Moser \cite{Moser1965}. This second proof is the one we will follow in Section~\ref{sec: darboux}. This proof requires some more knowledge of the geometry of differential forms and their associated operator. Moreover, even to make the counting argument work for symplectic forms, we would need this extra knowledge. For now, we will simply state the theorem and prove it after going into the prerequisites of differential geometry.
	\begin{restatable}{theorem}{darboux}\label{thm: darboux}
		Around every point on a symplectic manifold \(\h{\m,\omega}\), there exists a neighbourhood \(\h{U,\h{x^1,\ldots,x^n,y^1,\ldots,y^n}}\) such that on \(U\) the symplectic form is given by
		\begin{equation*}
			\omega|_U = \sum_{i = 1}^ndx^i\wedge dy^i.
		\end{equation*}
		Such coordinates are called Darboux coordinates.
	\end{restatable}
	\documentclass[class = report, crop = false]{standalone}
\usepackage{standalone}


\begin{document}	
\subsection{Operators on Differential Forms}
	Before we can prove Darboux's theorem, we will delve a bit deeper into the structure of differential forms and operators on them. In this section we will prove general results for \(k\)-forms, which are of course all applicable to symplectic forms as they are a special case with \(k = 2\). We will apply much of this theory in the proof of Darboux's theorem. Specifically, we will first discuss three operators: interior multiplication, exterior derivative and Lie derivative. Afterwards, we look into their connection in the form of Cartan's magic formula and prove Poincaré's lemma, which gives information on the local structure of closed differential forms. The definition and proofs will mainly be taken from \cite{Lee2013}, but the proof of Cartan's magic formula takes some insights from \cite{Marcut2017} and the proof of Poincaré's lemma is an adaptation of the proof of Corollary 13.2.14 in \cite{Marcut2017}.
	
	Firstly, we remind ourselves of the definition of \(k\)-forms. These stem from linear algebra, namely the exterior algebra. A \(k\)-form on a manifold can be interpreted as putting an alternating covariant \(k\)-tensor at each point in a smooth manner.
	\begin{definition}\label{def: linear alternating k-form}
		For a real vector space \(V\), we call a map \(\omega:\underbrace{V\times\cdots\times V}_{k \st[-times]}\to\R\) an \textbf{alternating \(k\)-form} if it is multilinear,
		\begin{equation*}
			\omega\h{v_1,\cdots,av_i + \tilde{a}\tilde{v}_i,\ldots,v_k} = a\omega\h{v_1,\cdots,v_i,\ldots,v_k} + \tilde{a}\omega\h{v_1,\cdots,\tilde{v}_i,\ldots,v_k}
		\end{equation*}
		and it is alternating
		\begin{equation*}
			\omega\h{v_{\sigma\h{1}},\ldots,v_{\sigma\h{k}}} = \sgn\h{\sigma}\omega\h{v_1,\ldots,v_k}.
		\end{equation*}
		We denote the space of alternating \(k\)-forms on \(V\) as \(\bigwedge^k\dual{V}\).
		
		On this space we define the\textbf{ wedge product} \(\wedge:\bigwedge^k\dual{V}\times\bigwedge^l\dual{V}\to\bigwedge^{k + l}\dual{V}\) as
		\begin{equation*}
			\h{\alpha\wedge\beta}\h{v_1,\ldots,v_{k + l}} = \dfrac{1}{k!l!}\sum_{\sigma\in S\h{k+l}}\sgn\h{\sigma}\alpha\h{v_{\sigma\h{1}}\ldots,v_{\sigma\h{k}}}\beta\h{v_{\sigma\h{k + 1}},\ldots,v_{\sigma\h{k + l}}},
		\end{equation*}
		where \(S\h{n}\) is the permutation group on \(n\) elements.
	\end{definition}
	\begin{example}\label{exp: wedge product}
		Suppose that \(V\) is a real vector space and \(\alpha,\beta\in\bigwedge^1\dual{V}\). The action of \(\alpha\wedge\beta\) on \(v_1,v_2\in V\) is then given by
		\begin{align*}
			\h{\alpha\wedge\beta}\h{v_1,v_2}
			&= \dfrac{1}{1!1!}\sum_{\sigma\in S\h{2}}\sgn\h{\sigma}\alpha\h{v_{\sigma\h{1}}}\beta\h{v_{\sigma\h{2}}}\\
			&= \alpha\h{v_1}\beta\h{v_2} - \alpha\h{v_2}\beta\h{v_1} = \h{\alpha\otimes\beta - \beta\otimes\alpha}\h{v_1,v_2}.
		\end{align*}
		Hence, the wedge product of two \(1\)-forms is given by \(\alpha\wedge\beta = \alpha\otimes\beta - \beta\otimes\alpha\).
	\end{example}
	As we can generate \(\h{k + l}\)-forms from a \(k\)-form and an \(l\)-form, we would want this operator to be one of a single space. To do this we define a more general algebraic structure.
	\begin{definition}\label{def: exterior algebra}
		The \textbf{exterior algebra of \(\dual{V}\)} is defined as 
		\begin{equation*}
			\bigwedge\dual{V} = \bigoplus_{k = 0}^n\bigwedge^k\dual{V}.
		\end{equation*}
		Furthermore, we extend the wedge product to a binary operation \(\wedge:\bigwedge\dual{V}\times\bigwedge\dual{V}\to\bigwedge\dual{V}\) using a bilinear extension of \(\wedge:\bigwedge^k\dual{V}\times\bigwedge^l\dual{V}\to\bigwedge^{k + l}\dual{V}\). In other words, we decompose  \(\alpha,\beta\in\bigwedge\dual{V}\) as
		\begin{equation*}
			\alpha = \alpha_0 + \cdots + \alpha_k,\qquad \alpha_i\in\bigwedge^i\dual{V},\qquad \beta = \beta_0 + \cdots + \beta_l,\qquad \beta_i\in\bigwedge^i\dual{V}.
		\end{equation*}
		Such that \(\alpha\wedge\beta\) is then given by
		\begin{equation}\label{eq: wedge product}
			\alpha\wedge\beta = \sum_{n = 0}^{k + l}\sum_{i + j = n}\alpha_i\wedge\beta_j.
		\end{equation}
		This defines the exterior algebra \(\h{\bigwedge\dual{V},+,\wedge}\).
	\end{definition}
	\begin{proposition}\label{prp: linear graded algebra}
		The triplet \(\h{\bigwedge\dual{V},+,\wedge}\) is an graded-commutative graded associative unital algebra. Meaning that \(\bigwedge^k\dual{V}\wedge\bigwedge^l\dual{V}\subset\bigwedge^{k + l}\dual{V}\) and for \(\alpha\in\bigwedge^k\dual{V},\beta\in\bigwedge^l\dual{V}\) the wedge product satisfies \(\alpha\wedge\beta = \h{-1}^{kl}\beta\wedge\alpha\).
	\end{proposition}
	\begin{proof}
		The fact that it is graded-commutative, and associative is a direct consequence of the definition of the wedge product as in Equation~\ref{eq: wedge product}. The unit is \(1\in\R\), hence, it is unital.
	\end{proof}
	We can extend these alternating \(k\)-forms on real vector spaces to differential forms on manifolds by defining them pointwise.
	\begin{definition}\label{def: alternating k-form}
		An \textbf{alternating \(k\)-tensor} is an element of the set
		\begin{equation*}
			\bigwedge^k\cotang = \coprod_{p\in\m}\bigwedge^k\loccotang{p}.
		\end{equation*}
		A \textbf{differential \(k\)-form}, or simply\textbf{ \(k\)-form}, is then a smooth section of \(\bigwedge^k\cotang\), where we denote the set of \(k\)-forms by	
		\begin{equation*}
			\Omega^k\h{\m} = \Gamma\h{\bigwedge^k\cotang}.
		\end{equation*}
		This naturally forms a real vector space with pointwise addition and scalar multiplication. We define the \textbf{wedge product} \(\wedge:\Omega^k\h{\m}\times\Omega^l\h{\m}\to\Omega^{k + l}\h{\m}\) pointwise by \(\h{\omega\wedge\eta}_p = \omega_p\wedge\eta_p\). Leading to the definition of the algebra
		\begin{equation*}
			\Omega\h{\m} = \bigoplus_{k = 0}^n\Omega^k\h{\m}.
		\end{equation*}
		This gives us the algebra \(\h{\Omega\h{\m},+,\wedge}\), where \(\wedge\) is the bilinear extension of the wedge product above as in Definition \ref{def: exterior algebra}
	\end{definition}
	\begin{proposition}\label{prp: graded algebra}
		The algebra \(\h{\Omega\h{\m},+,\wedge}\) is a graded-commutative graded associative unital algebra.
	\end{proposition}
	\begin{proof}
		This is a direct consequence of Proposition~\ref{prp: linear graded algebra}. In this case, the unit is given by the function\(f:\m\to1\in\sff\), hence, it is a unital algebra.
	\end{proof}
	In any smooth chart \(\h{U,\h{x^i}}\) we can write an \(\omega\in\Omega^k\h{\m}\) as
	\begin{equation*}
		\omega = \sum_{i_1<\cdots<i_k}\omega_{i_1\ldots i_k}dx^{i_1}\wedge\cdots\wedge dx^{i_k},
	\end{equation*}
	where \(\omega_{i_1,\ldots i_k}\in\sff\). This notation is called ordered indices, but there are more possible notations. Most importantly
	\begin{equation*}
		\omega = \sum_{i_1\ldots i_k}\dfrac{1}{k!}\omega_{i_1\ldots i_k}dx^i\wedge\cdots\wedge dx^{i_k},\qquad \omega_{\sigma\h{i_1}\ldots\sigma\h{i_k}} = \sgn\h{\sigma}\omega_{i_1\ldots i_k}\in\sff.
	\end{equation*}
	This is with unordered indices, which can be useful sometimes. Lastly, we may abbreviate our notation for conciseness to
	\begin{equation*}
		\omega = \sum_I\omega_Idx^I.
	\end{equation*}
	Here \(\omega_I\in\sff\) and \(dx^I\) is an abbreviation for \(dx^{i_1}\wedge\cdots\wedge dx^{i_k}\). All of these forms are equivalent, but some may be more convenient to use. They can always be identified by how the indices in the sum are written.
	
	With the construction of the differential forms and structure of \(\Omega\h{\m}\) more clear, we will now look at some different operators.
	
	\subsubsection{Interior multiplication}
		The first operation we are interested in is interior multiplication. We will define this on the exterior algebra of a real vector space, which is then naturally extended to differential forms. The interior multiplication is defined such that it fixes the first argument of a \(k\)-form.
		\begin{definition}\label{def: linear interior multiplication}
			For an \(\omega\in\bigwedge^k\dual{V}\) and \(v\in V\) we define the \textbf{interior multiplication of \(\omega\) by \(v\)}, denoted as \(\iota_v\omega\), by fixing the first argument in \(\omega\). This is more clear by considering its action on some \(v_1,\ldots,v_{k - 1}\in V\), then
			\begin{equation*}
				\iota_v\omega\h{v_1,\ldots,v_{k - 1}} = \omega\h{v,v_1,\ldots,v_{k - 1}},
			\end{equation*}
			and \(\iota_v\omega = 0\) if \(k = 0\). The associated operator \(\iota_v:\bigwedge^\star\dual{V}\to\bigwedge^{\star - 1}\dual{V}\) is called the \textbf{interior multiplication by \(v\)} or \textbf{contraction by \(v\)}.
		\end{definition}
		The most important property of this operator is that it respects the structure of the exterior algebra and vector space.
		\begin{proposition}\label{prp: linear contraction linearity}
			The interior multiplication is \(\R\)-linear in \(v\) and \(\omega\), i.e. \(\iota:V\times\bigwedge^k\dual{V}\to\bigwedge^{k - 1}\dual{V}:\h{v,\omega}\mapsto\iota_v\omega\) is bilinear.
		\end{proposition}
		\begin{proof}
			The linearity in \(v\) follows from the definition, combined with the multilinearity of \(k\)-forms. Suppose that \(\omega\in\bigwedge^k\dual{V}\), \(v,\tilde{v},v_1,\ldots,v_{k - 1}\in V\) and \(a,\tilde{a}\in\R\), then
			\begin{align*}
				\iota_{av + \tilde{a}\tilde{v}}\h{\omega}\h{v_1,\ldots,v_{k - 1}}
				&= \omega\h{av + \tilde{a}\tilde{v},v_1,\ldots,v_{k - 1}}\\
				&= a\omega\h{v,v_1,\ldots,v_{k - 1}} + \tilde{a}\omega\h{\tilde{v},v_1,\ldots,v_{k - 1}}\\
				&= a\iota_v\h{\omega}\h{v_1,\ldots,v_{k - 1}} + \tilde{a}\iota_{\tilde{v}}\h{\omega}\h{v_1,\ldots,v_{k - 1}}.
			\end{align*}
			Thus, \(\iota\) is linear in its first component. For the second component, we use the definition of addition and scalar multiplication of tensor fields. Hence, for \(\omega,\tilde{\omega}\in\bigwedge^k\dual{V}\), \(v,v_1,\ldots,v_{k - 1}\in V\) and \(a,\tilde{a}\in\R\) we have
			\begin{align*}
				\iota_v\h{a\omega + \tilde{a}\tilde{\omega}}\h{v_1,\ldots,v_{k - 1}}
				&= \h{a\omega + \tilde{a}\tilde{\omega}}\h{v,v_1,\ldots,v_{k - 1}}\\
				&= a\omega\h{v,v_1,\ldots,v_{k - 1}} + \tilde{a}\tilde{\omega}\h{v,v_1,\ldots,v_{k - 1}}\\
				&= a\iota_v\h{\omega}\h{v_1,\ldots,v_{k - 1}} + \tilde{a}\iota_v\h{\tilde{\omega}}\h{v_1,\ldots,v_{k - 1}}.
			\end{align*}
			Hence, \(\iota\) is also linear in its second component.
		\end{proof}
		Besides respecting the vector space structure, it also works with the graded-commutative graded algebra structure of \(\bigwedge\dual{V}\).
		\begin{proposition}\label{prp: linear contraction anti-derivation}
			The interior multiplication is a graded derivation of degree \(-1\). In other words, \(\iota_v:\bigwedge^{\star}\dual{V}\to\bigwedge^{\star-1}\dual{V}\) satisfies
			\begin{equation}\label{eq: interior multiplication product rule}
				\iota_v\h{\omega\wedge\eta} = \iota_v\omega\wedge\eta + \h{-1}^k\omega\wedge\iota_v\eta,
			\end{equation}
			where \(\omega\in\bigwedge^k\dual{V}\) and \(\eta\in\bigwedge^l\dual{V}\).
		\end{proposition}
		\begin{proof}
			Take some \(v\) from a vector space \(V\). As \(\iota_v\) is \(\R\)-linear as a mapping \(\iota_v:\bigwedge^{\star}\dual{V}\to\bigwedge^{\star - 1}\dual{V}\), see Proposition~\ref{prp: linear contraction linearity}, we may assume \(\omega = \alpha^1\wedge\cdots\wedge\alpha^k\) and \(\eta = \alpha^{k + 1}\wedge\cdots\wedge\alpha^{k + l}\), where \(\alpha^i\in\bigwedge^1\dual{V}\). We can then prove Equation \ref{eq: interior multiplication product rule} by evaluating it in some vectors \(v_1\ldots,v_{k + l}\in V\) and using the inductive definition of the determinant, see \cite[Proposition 14.11 (e)]{Lee2013}, and splitting the sum,
			\begin{align*}
				\iota_v\h{\omega\wedge\eta}
				&= \iota_v\h{\alpha^1\wedge\cdots\wedge\alpha^{k + l}} = \sum_{i = 1}^{k + l}\h{-1}^{i - 1}\alpha^i\h{v}\alpha^1\wedge\cdots\wedge\widehat{\alpha}^i\wedge\cdots\wedge\alpha^{k + l}\\
				&= \h{\sum_{i = 1}^k\h{-1}^{i - 1}\alpha^i\h{v}\alpha^1\wedge\cdots\wedge\widehat{\alpha}^i\wedge\cdots\wedge\alpha^{k}}\wedge\alpha^{k + 1}\wedge\cdots\wedge\alpha^{k + l}\\
				&\qquad + \h{-1}^k\alpha^1\wedge\cdots\wedge\alpha^{k}\h{\sum_{i = 1}^{l}\h{-1}^{i - 1}\alpha^{k + i}\h{v}\alpha^{k + 1}\wedge\cdots\wedge\widehat{\alpha}^{k + i}\wedge\cdots\wedge\alpha^{k + l}}\\
				&= \iota_v\omega\wedge\eta + \h{-1}^k\omega\wedge\iota_v\eta.
			\end{align*}
			As we mentioned, this is enough to prove that \(\iota_v\) is an anti-derivation of degree \(-1\).
		\end{proof}
		We can quite naturally extend the interior multiplication by recognising that a differential form is an element of the exterior algebra of the tangent spaces at each point. Hence, the interior multiplication can be implemented locally. To do this we should remark that we will need a vector at each point to contract with, enticing us to make use of a vector field instead of a single vector.
		\begin{definition}\label{def: interior multiplication}
			Let \(\omega\) be a \(k\)-form and \(X\) a vector field, both on a manifold \(\m\). We denote the \textbf{interior multiplication of \(\omega\) by \(X\)} as \(\iota_X\omega\) and define it for a point \(p\in\m\) as
			\begin{equation*}
				\h{\iota_X\omega}_p = \iota_{X_p}\omega_p.
			\end{equation*}
			Here the right-hand side is the interior multiplication on the exterior algebra as in Definition~\ref{def: linear interior multiplication}. Similar to the linear case, we call \(\iota_X:\Omega^\star\h{\m}\to\Omega^{\star - 1}\h{\m}\) the \textbf{interior multiplication with \(X\)} or \textbf{contraction by \(X\)}.
		\end{definition}
		As this extension is done pointwise, it still holds the same properties as the interior multiplication on the exterior algebra.
		\begin{proposition}\label{prp: contraction linearity}
			The mapping \(\iota:\vf\times\Omega^k\h{\m}\to\Omega^{k - 1}\h{\m}:\h{X,\omega}\mapsto\iota_X\omega\) is \(\R\)-linear in both components.
		\end{proposition}
		\begin{proof}
			This is a direct consequence of the definition of the interior multiplication, see Definition~\ref{def: interior multiplication}, and Proposition~\ref{prp: linear contraction linearity}. We should check that for a \(k\)-form \(\omega\) the form \(\iota_X\omega\) is indeed a smooth section.
			
			Suppose that \(\m\) is a manifold and \(X\in\vf\). Take some \(\omega\in\Omega^k\h{\m}\) and a chart \(\h{U,\h{x^i}}\). We can then write \(\omega\) locally as
			\begin{equation*}
				\omega|_U = \sum_{i_1\ldots i_k}\dfrac{1}{k!}\omega_{i_1\ldots i_k}dx^i\wedge\cdots\wedge dx^{i_k},\quad \omega_{\sigma\h{i_1}\ldots\sigma\h{i_k}} = \sgn\h{\sigma}\omega_{i_1\ldots i_k}.
			\end{equation*}
			As the interior multiplication is defined pointwise, it follows that \(\h{\iota_X\omega}|_U = \iota_X\h{\omega|_U}\). Hence, we can express \(\h{\iota_X\omega}|_U\) as
			\begin{equation}\label{eq: action of interior}
				\h{\iota_X\omega}|_U = \sum_{i_1\ldots i_k}\dfrac{1}{\h{k - 1}!}X^{i_1}\omega_{i_1\ldots i_k}dx^{i_2}\wedge\cdots\wedge dx^{i_k},\quad \omega_{\sigma\h{i_1}\ldots \sigma\h{i_k}} = \sgn\h{\sigma}\omega_{i_1\ldots i_k}.
			\end{equation}
			As \(X\in\vf\) and \(\omega\in\Omega^k\h{\m}\), it follows that \(X^{i_1}\) and \(\omega_{i_1\ldots i_k}\) are both smooth functions. Therefore \(\iota_X\omega\) has smooth coordinate functions, implying it is a smooth section of \(\bigwedge^{k - 1}\cotang\), i.e. it is a \(\h{k - 1}\)-form.
		\end{proof}
		\begin{proposition}\label{prp: contraction anti-derivation}
			The interior multiplication is a graded-derivation of degree \(-1\). In other words, it is a mapping \(\iota_X:\Omega^{\star}\h{\m}\to\Omega^{\star - 1}\h{\m}\) such that 
			\[
				\iota_X\h{\omega\wedge\eta} = \iota_X\omega\wedge\eta + \h{-1}^k\omega\wedge\iota_X\eta,
			\]
			where \(\omega\in\Omega^k\h{\m}\) and \(\eta\in\omega^l\h{\m}\).
		\end{proposition}
		\begin{proof}
			This all follows from the fact that the interior multiplication is defined pointwise in combination with Proposition~\ref{prp: linear contraction anti-derivation}.
		\end{proof}
		
	\subsubsection{Exterior derivative}\label{sec: exterior derivative}
		The second operator we introduce is the exterior derivative. This forms the extension of the differential on a function. The definition follows from the fact that a \(1\)-form that is exact, meaning there exists an \(f\in\sff\) such that \(\omega = df\), necessarily is closed, meaning \(\pdv*{\omega_j}{x^i} - \pdv*{\omega_i}{x^j} = 0\) in any chart. We will first introduce the exterior derivative on a Euclidean space, after which it can easily be generalised to a manifold.
		\begin{definition}\label{def: linear exterior derivative}
			For a \(k\)-form \(\omega = \sum_I\omega_Idx^I\) on \(\R[n]\) we define \textbf{the exterior derivative of \(\omega\)}, denoted as \(d\omega\), using the formula
			\begin{equation*}
				d\omega = d\h{\sum_I\omega_Idx^I} = \sum_Id\omega_I\wedge dx^I,
			\end{equation*}
			where \(d\omega_I\) is defined as the differential of a function. The associated operator \(d:\Omega^\star\h{\m}\to\Omega^{\star + 1}\h{\m}\) is called the \textbf{exterior derivative on \(\R[n]\)}.
		\end{definition}
		In the definition, we used the shorthand notation as it gives a clean formula. For calculations, the ordered notation will prove more useful. Translating the exterior derivative to this notation, we get the expression
		\begin{equation*}
			d\omega = \sum_{i_1<\cdots<i_k}\sum_i\pdv{\omega_{i_1\ldots i_k}}{x^i}dx^i\wedge dx^{i_1}\wedge\cdots\wedge dx^{i_k}.
		\end{equation*}
		Again this operator acts as a form of derivation on \(\Omega^\star\h{\R[n]}\).
		\begin{proposition}\label{prp: linear exterior derivative properties}
			The exterior derivative on \(\R[n]\) has the following properties:
			\begin{enumerate}[label = ({\arabic*)}, ref={\theproposition(\arabic*)}]
				\item\label{part: linear exterior derivative linearity} \(d\) is linear over \(\R\).
				\item\label{part: linear exterior derivative differential}For an \(f\in C^\infty\h{\R[n]}\) and \(X\in\vf[{\R[n]}]\) is satisfies \(df\h{X} = Xf\).
				\item\label{part: linear exterior derivative anti-derivation} It is a graded derivation, i.e. for any \(\omega\in\Omega^k\h{\R[n]}\) and \(\eta\in\Omega^l\h{\R[n]}\) we have
				\begin{equation*}
					d\h{\omega\wedge\eta} = d\omega\wedge\eta + \h{-1}^k\omega\wedge d\eta.
				\end{equation*}
				\item\label{part: linear exterior derivative square} Its square is zero, i.e. \(d\circ d\h{\omega} = 0\) for any \(\omega\in\Omega^k\h{\R[n]}\).
			\end{enumerate}
			More concisely, we might call it a graded derivation of the graded algebra \(\Omega\h{\R[n]}\) of degree \(+1\) with a vanishing square, which coincides with the differential on functions.
		\end{proposition}
		\begin{proof}
			The linearity of \(d\) follows from the definition, the fact that \(d\) is linear on smooth functions and that \(\wedge\) is distributive over addition.
			
			The second property follows directly from the definition. Take an arbitrary \(f\in C^\infty\h{\R[n]}\) and \(X\in\vf[{\R[n]}]\), a straightforward calculation of the action of \(df\) on \(X\) shows the second property.
			\begin{equation*}
				df\h{X} = \pdv{f}{x^i}dx^i\h{X} = X^i\pdv{f}{x^i} = Xf.
			\end{equation*}
			
			To prove the third property, we will use the first property much like we did in the proof of Proposition~\ref{prp: contraction linearity}. Therefore, we only consider terms of the form \(\omega = \omega_Idx^I\in\Omega^k\h{\R[n]}\) and \(\eta = \eta_Jdx^J\in\Omega^l\h{\R[n]}\). By the fact that the exterior derivative is the differential on functions and therefore satisfies the Leibniz rule, it follows that,
			\begin{align*}
				d\h{\omega\wedge\eta}
				&= d\h{\omega_I\eta_J dx^{IJ}} = d\h{\omega_I\eta_J}dx^{IJ} = \h{\h{d\omega_I}\eta_J + \omega_I\h{d\eta_J}}\wedge dx^{IJ},\\
				&= \h{d\omega_I\wedge dx^I}\wedge\eta_Jdx^J - \h{-1}^k\omega_Idx^I\wedge\h{d\eta_J\wedge dx^J} = d\omega\wedge\eta + \h{-1}^k\omega\wedge d\eta.
			\end{align*}
			As we mentioned, by the \(\R\)-linearity of \(d\) this implies that \(d\) satisfies the product rule with respect to \(\wedge\).
			
			To prove the last property, we will first consider the action of the exterior derivative on a function. For an arbitrary \(f\in\Omega^0\h{\R[n]} = C^\infty\h{\R[n]}\) we use the fact that the second order partial differential is symmetric such that
			\begin{align*}
				d\h{df}
				&= d\h{\pdv{f}{x^i}dx^i} = d\h{\pdv{f}{x^i}}\wedge dx^i = \pdv{f}{x^i}{x^j}dx^j\wedge dx^i,\\
				&= \sum_{i < j}\h{\pdv{f}{x^i}{x^j} - \pdv{f}{x^j}{x^i}}dx^i\wedge dx^j = 0.
			\end{align*}
			Hence \(d\circ d = 0\) on functions, luckily we can reduce the case on arbitrary \(k\)-forms to that of functions as follows using
			\begin{align*}
				d\h{d\omega}
				&= d\h{d\omega_{i_1\ldots i_k}\wedge dx^{i_1}\wedge\cdots\wedge dx^{i_k}},\\
				&= d\h{d\omega_{i_1\ldots i_k}}\wedge dx^{i_1}\wedge\cdots\wedge dx^{i_k} + d\omega_{i_1\ldots i_k}\wedge d\h{dx^{i_1}\wedge\cdots\wedge dx^{i_k}}.
			\end{align*}
			The first term is zero as we showed that \(d\circ d\) is zero on functions. The second term can also be shown to be zero by using the following inductive argument
			\begin{equation*}
				d\h{dx^{i_1}\wedge\cdots\wedge dx^{i_k}} = d\h{dx^{i_1}}\wedge\cdots\wedge dx^{i_k} - dx^{i_1}\wedge d\h{dx^{i_2}\wedge\cdots\wedge dx^{i_k}}.
			\end{equation*}
			The first term is zero by the same argument as before, the second term is zero due to the inductive argument. Hence, we have shown that \(d\h{d\omega} = 0\) for an arbitrary \(\omega\).
		\end{proof}
		Besides these algebraic properties, the exterior derivative also works naturally with smooth functions between open subsets.
		\begin{proposition}\label{prp: linear exterior derivative pullback}
			For any \(U\subset \R[n]\), \(V\subset\R[m]\) both open and function \(F:V\to U\) we have
			\begin{equation*}
				\pull{F}\h{d\omega} = d\h{\pull{F}\omega},
			\end{equation*}
			where \(\omega\) is an arbitrary \(k\)-form on \(V\).
		\end{proposition}
		\begin{proof}
			We can use the linearity of the exterior derivative, see Proposition~\ref{part: linear exterior derivative linearity}, such that it suffices to show this for \(\omega = \omega_{i_1\ldots i_k}dx^{i_1}\wedge\cdots\wedge dx^{i_k}\). Suppose that \(F:V\to U\), where \(U\subset\R[n]\) and \(V\subset\R[m]\) and \(\omega\in\Omega^k\h{U}\), then
			\begin{align*}
				\pull{F}\h{d\omega}
				&=  d\h{\omega_{i_1\ldots i_k}\circ F}\wedge d\h{x^{i_1}\circ F}\wedge \cdots \wedge d\h{x^{i_k}\circ F},\\
				&= d\h{\h{\omega_{i_1\ldots i_k}\circ F}\wedge d\h{x^{i_1}\circ F}\wedge \cdots \wedge d\h{x^{i_k}\circ F}} = d\h{\pull{F}\omega}.
			\end{align*}
			This proves our result.
		\end{proof}
		We can use the properties of the exterior derivative on \(\R[n]\), see Propositions~\ref{prp: linear exterior derivative properties}~and~\ref{prp: linear exterior derivative pullback}, to extend our definition to an arbitrary manifold.
		\begin{proposition}\label{prp: existence exterior derivative manifold}
			For a manifold \(\m\), there exists a unique \(\R\)-linear operator \(d:\Omega^k\h{\m}\to\Omega^{k + 1}\h{\m}\) for any \(k\) which satisfies the following:
			\begin{enumerate}[ref = \theproposition.(\arabic*)]
				\item\label{part: exterior derivative differential} For a function \(f\in \sff\) it is defined as the differential.
				\item\label{part: exterior derivative anti-derivation} It is a graded derivation, i.e. for any \(\omega\in\Omega^k\h{\m}\) and \(\eta\in\omega^l\h{\m}\) we have
				\begin{equation*}
					d\h{\omega\wedge\eta} = d\omega\wedge\eta + \h{-1}^k\omega\wedge d\eta.
				\end{equation*}
				\item\label{part: exterior derivative square} Its square is zero, i.e. \(d\circ d\h{\omega} = 0\) for any \(\omega\in\Omega^k\h{\m}\).
			\end{enumerate}
		\end{proposition}
		\begin{proof}
			Much like any unique existence theorem, we will first prove the existence and then prove that this is also unique. The existence can be derived locally from the definition of the exterior derivative on \(\R[n]\). The uniqueness is a consequence of the product rule.
			
			Suppose that \(\m\) is a manifold and \(\omega\in\Omega^k\h{\m}\) for some arbitrary \(k\), then let \(\h{U,\phi}\) be a chart on \(\m\). We define \(d\omega\) locally on \(U\) as
			\begin{equation*}
				\h{d\omega}|_U = \h{\pull{\phi}\circ d\circ\pull{\h{\phi^{-1}}}}\h{\omega|_U}.
			\end{equation*}
			Remark that the \(d\) on the right-hand side is the exterior derivative on \(\R[n]\).
			
			As this defines the exterior derivative in a specific chart, we should now check the value does not depend on our choice of chart. Let \(\h{V,\psi}\) be another chart on \(\m\), we will then calculate \(d\omega|_{U\cap V}\) and show that it is invariant under coordinate transformations by using Proposition~\ref{prp: linear exterior derivative pullback} on \(\phi\circ\psi^{-1}:\psi\h{U\cap V}\to \phi\h{U\cap V}\),
			\begin{align*}
				\h{\pull{\phi}\circ d\circ\pull{\h{\phi^{-1}}}}\h{\omega|_{U\cap V}}
				&= \h{\pull{\psi}\circ\pull{\h{\psi^{-1}}}\circ\pull{\phi}\circ d\circ\pull{\h{\phi^{-1}}}}\h{\omega|_{U\cap V}},\\
				&= \h{\pull{\psi}\circ d\circ\pull{\h{\psi^{-1}}}\circ\pull{\phi}\circ\pull{\h{\phi^{-1}}}}\h{\omega|_{U\cap V}},\\
				&= \h{\pull{\psi}\circ d\circ\pull{\h{\psi^{-1}}}}\h{\omega|_{U\cap V}}.
			\end{align*}
			Thus we have shown that the exterior derivative is independent of the choice of a coordinate chart, which implies that it is well-defined globally. It is then clear from Proposition~\ref{prp: linear exterior derivative properties} that it satisfies the properties mentioned locally and by construction everywhere.
			
			For uniqueness, we will first show that any operator which is a graded derivation is a local operator, i.e. if \(\eta\in\Omega^k\h{\m}\) satisfy \(\eta|_U = 0\) for some open \(U\subset\m\), then \(d\eta|_U = 0\). This then lets us prove the uniqueness locally.
			
			Suppose that \(D\) is a graded derivation on \(\Omega^k\h{\m}\), and let \(\eta\) be some \(k\)-form for which there exists an open \(U\subset \m\) such that \(\eta|_U = 0\). Take a \(p\in U\) and take a bump function \(\psi\) such that there exists an open neighbourhood \(V\) of \(p\) for which \(\psi|_V = 1\) and \(\supp\h{\psi}\subset U\). We can then conclude from the fact that \(\eta|_{\supp\h{\psi}} = 0\) that \(\psi\eta = 0\). Hence, it follows with \(\eta_p = 0\) and \(\psi\h{p} = 1\) that 
			\begin{equation*}
				0 = D\h{\psi\eta}_p = \h{D\psi}_p\wedge\eta_p + \psi\h{p}\wedge \h{D\eta} = \h{D\eta}_p.
			\end{equation*}
			This shows that \(D\eta|_U = 0\) if \(\eta|_U = 0\), and hence \(D\) is a local operator. Remark that this specifically applies to the exterior derivative.
			
			Now suppose that \(D:\Omega^k\h{\m} \to\Omega^{k + 1}\h{\m}\) is an operator which satisfies the properties of the proposition. We will then show that \(D\) coincides with \(d\) by first proving this for smooth functions, and then showing that \(D\) vanishes on products of differentials. It should be clear that for any \(f\in\sff\) we have \(Df = df\) as they are both equal to the differential of \(f\). Suppose that \(f^1,\ldots,f^k\in\sff\), it follows from the properties of the proposition that
			\begin{equation*}
				D\h{df^1\wedge \cdots \wedge df^k} = D\h{Df^1\wedge \cdots \wedge Df^k} = \sum_{i = 1}^k\h{-1}^{i - 1}Df^1\wedge\cdots\wedge D^2f^{i}\wedge\cdots\wedge Df^k = 0.
			\end{equation*}
			Take some arbitrary \(\eta\in\Omega^k\h{\m}\) and \(p\in\m\), and let \(\h{U,\h{x^i}}\) be some coordinates around \(p\). Construct some smooth bump function \(\psi\) such that there exists an open neighbourhood \(V\) of \(p\) such that \(\psi|_V = 1\) and \(\supp\h{\psi}\subset U\). Furthermore, in these coordinates we can write \(\eta\) as \(\eta|_U = \sum_I\eta_I dx^I\). Extend the functions \(x^i\) and \(\eta_I\) smoothly using the smooth bump function to functions \(\tilde{x}^i\) and \(\tilde{\eta}_I\) such that \(\tilde{x}^i|_V = x^i|_V\) and \(\tilde{\eta}_I|_V = \eta_I|_V\). We can then define an extension of \(\eta\) as \(\tilde{\eta} = \sum_I\tilde{\eta}_Id\tilde{x}^I\) such that it is defined on the whole of \(\m\) and satisfies \(\tilde{\eta}|_V = \eta|_V\). By the locality of \(D\) and \(d\) it follows that \(\h{D\eta}|_V = \h{D\tilde{\eta}}|_V\) and \(\h{d\eta}|_V = \h{d\tilde{\eta}}|_V\). As \(p\in V\), we can conclude from the discussion above that
			\begin{align*}
				\h{D\eta}_p
				&= \h{D\tilde{\eta}}_p = \h{D\sum_I\tilde{\eta}_Id\tilde{x}^I}_p = \h{\sum_ID\tilde{\eta}_I\wedge d\tilde{x}^I + \tilde{\omega}_I\wedge Dd\tilde{x}^I}_p\\
				&= \h{\sum_id\tilde{\eta}_I\wedge d\tilde{x}^I}_p = \h{d\tilde{\eta}}_p = \h{d\eta}_p.
			\end{align*}
 		As our choice of \(p\) is arbitrary, it follows that \(D = d\) on the whole manifold and thus that \(d\) is uniquely defined.
		\end{proof}
		\begin{definition}\label{def: exterior derivative}
			The unique operator \(d\) defined in Proposition~\ref{prp: existence exterior derivative manifold} is called the \textbf{exterior derivative on \(\m\)}. It is the unique linear operator that is a graded derivation on \(\Omega\h{\m}\) of degree \(+1\) with a vanishing square that coincides with the differential on smooth functions.
		\end{definition}
		The exterior derivative lets us define special classes of differential forms, namely closed and exact differential forms. These play a much bigger role in the study of the de Rham cohomology groups.
		\begin{definition}
			A differential form \(\omega\in\Omega^k\h{\m}\) is called \textbf{closed} if \(d\omega = 0\), and \textbf{exact} if there exists an \(\alpha\in\Omega^{k - 1}\h{\m}\) such that \(\eta = d\alpha\). Let \(Z^k\h{\m}\) denoted the set of closed \(k\)-forms on \(\m\) and \(B^k\h{\m}\) the set of exact \(k\)-forms, remark that \(Z^k\h{\m} = \ker\h{d:\Omega^k\h{\m}\to\Omega^{k + 1}\h{\m}}\) and \(B^k\h{\m} = \Im\h{d:\Omega^{k - 1}\h{\m}\to\Omega^k\h{\m}}\).
		\end{definition}
		We will now show a similar statement to Proposition~\ref{prp: linear exterior derivative pullback}, proving that the exterior derivative on an arbitrary manifold is natural in the sense that it commutes with the pullback.
		\begin{proposition}\label{prp: exterior derivative pullback}
			For a smooth function \(F:\m\to\m[N]\) and an arbitrary \(\omega\in\Omega^k\h{\m[N]}\), we have that
			\begin{equation*}
				\pull{F}\h{d\omega} = d\h{\pull{F}\omega}.
			\end{equation*}
			In other words, the exterior derivative commutes with the pullback.
		\end{proposition}
		\begin{proof}
			This follows directly from applying Proposition~\ref{prp: linear exterior derivative pullback} to the coordinate representations of \(F\) and \(\omega\).
		\end{proof}
		A last important property of the differential is the fact that it can commute with integrals in a special manner that is reminiscent of the Leibniz integral rule.
		\begin{proposition}\label{prp: exterior derivative integral}
			Given a smooth family of differential forms \(\omega_t\in\Omega^k\h{\m}\), with \(t\in\ha{0,1}\), it satisfies the following
			\begin{equation*}
				\int_0^1\h{d\omega_t}dt = d\h{\int_0^1\omega_tdt}.
			\end{equation*}
			where \(d\) is the exterior derivative on \(\m\).
		\end{proposition}
		\begin{proof}
			We can assume that \(\m\) has some global coordinates \(\h{x^i}\). Next up, suppose that \(\omega_t\in\Omega^k\h{\m}\) is a smooth family of \(k\)-forms with \(t\in\ha{0,1}\). We can then use the fact that partial derivatives commute with integrals of constant boundaries.
			\begin{align*}
				\int_0^1\h{d\omega_t}dt
				&= \int_0^1\h{d\h{\h{\omega_t}_Idx^I}}dt = \int_0^1\h{\pdv{\h{\omega_t}_I}{x^i}dx^i\wedge dx^I}dt,\\
				&= \h{\int_0^1\pdv{\h{\omega_t}_I}{x^i}dt}dx^i\wedge dx^I =  \pdv{x^i}\h{\int_0^1\h{\omega_t}_Idt}dx^i\wedge dx^I,\\
				&= d\h{\int_0^1\h{\omega_t}_I dt\wedge dx^I} = d\h{\int_0^1\omega_t dt}.
			\end{align*}
			Which was what we wanted.
		\end{proof}
		
	\subsubsection{Lie derivative}\label{sec: lie derivative}
		The last operator we will introduce in this section is the Lie derivative of a differential form. This forms a natural extension of the Lie derivative on functions and vector fields, see Definitions~\ref{def: lie derivative function} and~\ref{def: lie derivative vector field}.
		\begin{definition}
			The \textbf{Lie derivative} of a differential form \(\omega\in\Omega^k\h{\m}\) along some \(X\in\vf\) is defined as
			\begin{equation*}
				\ld[X]\omega = \dtnull \pulltimeflow{t}\omega.
			\end{equation*}
			The existence of the derivative is ensured by Theorem~\ref{thm: flow domain}. We call the operator \(\ld[X]:\Omega^\star\h{\m}\to\Omega^\star\h{\m}\) the \textbf{Lie derivative along \(X\)}.
 		\end{definition}
 		As the name implies, this operator is a derivation on \(\Omega\h{\m}\).
 		\begin{proposition}\label{prp: lie derivative derivation}
 			The Lie derivative is a derivation of degree \(0\). In other words, for any \(X\in\vf\) it is a mapping \(\ld[X]:\Omega^{\star}\h{\m}\to\Omega^{\star}\h{\m}\) such that
 			\begin{equation*}
 				\ld[X]\h{\omega\wedge\eta} = \ld[X]\omega\wedge\eta + \omega\wedge\ld[X]\eta,
 			\end{equation*}
 			where \(\omega\in\Omega^k\h{\m}\) and \(\eta\in\Omega^l\h{\m}\).
 		\end{proposition}
 		\begin{proof}
			Let \(X\) be a vector field on \(\m\), \(\omega\) a \(k\)-form and \(\eta\) an \(l\)-form. Remark that the pullback distributes over the wedge product, see \cite[Lemma 14.16 (b)]{Lee2013}, i.e.
			\begin{equation*}
				\pull{\timeflow{t}}\h{\omega\wedge\eta} = \h{\pull{\timeflow{t}}\omega}\wedge\h{\pull{\timeflow{t}}\eta}.
			\end{equation*}
			Hence, we get
			\begin{align*}
				\ld[X]\h{\omega\wedge\eta}
				&= \dtnull\ha{\pulltimeflow{t}\h{\omega\wedge\eta}} = \dtnull\ha{\pulltimeflow{t}\omega\wedge\pulltimeflow{t}\eta},\\
				&= \ha{\dtnull\pulltimeflow{t}\omega}\wedge\eta + \omega\wedge\ha{\dtnull\pulltimeflow{t}\eta} = \ld[X]\omega\wedge\eta + \omega\wedge\ld[X]\eta.
			\end{align*}
			This proves the proposition.
 		\end{proof}
 		Now we can already make a simple connection between the Lie derivative and the exterior derivative.
 		\begin{proposition}\label{prp: exterior derivative and lie derivative commute}
 			The exterior derivative commutes with the Lie derivative.
 		\end{proposition}
 		\begin{proof}
 			Suppose \(X\) is some vector field on \(\m\) and \(\omega\in\Omega^k\h{\m}\). Using Proposition~\ref{prp: exterior derivative pullback}, we obtain our result
 			\begin{equation*}
 				\h{\ld[X]\circ d}\omega = \dtnull\pulltimeflow{t}\h{d\omega} = \dtnull d\h{\pulltimeflow{t}\omega} = \h{d\circ\ld[X]}\omega.
 			\end{equation*}
 			As \(\omega\) is arbitrary, we get \(\ld[X]\circ d = d\circ\ld[X]\).
 		\end{proof}
 		Besides this connection to the exterior derivative, the Lie derivative lets us have a natural connection with the flow along a vector field.
 		\begin{proposition}\label{prp: lie derivative flow commute}
 			Let \(X\) be a vector field on a manifold \(\m\). If \(\phi_X^t\) denotes the flow of \(X\) at time \(t\), then for any \(\alpha\in\Omega\h{\m}\)
 			\begin{equation*}
 				\eval{\dv{t}}_{t = t_0}\pulltimeflow{t}\alpha = \pulltimeflow{t_0}\h{\ld[X]\alpha}.
 			\end{equation*}
 		\end{proposition}
 		\begin{proof}
 			Let \(X\) be a vector field on a manifold \(\m\) and \(\alpha\in\Omega\h{\m}\), we can write out the definitions above for a point \(p\in\m\), where we need to ensure that \(\h{t_0,p}\in\mathcal{D}\h{X}\). It then follows by substituting \(t = s + t_0\) and using Proposition~\ref{prp: flow one-parameter group} and Proposition 12.25 (e) in \cite{Lee2013},
 			\begin{align*}
 				\eval{\dv{t}}_{t = t_0}\pulltimeflow{t}\alpha
 				&= \eval{\dv{s}}_{s = 0}\pulltimeflow{s + t_0}\alpha = \eval{\dv{s}}_{s = 0}\pulltimeflow{t_0}\pulltimeflow{s}\alpha,\\
 				&= \pull{\h{\pulltimeflow{t_0}}}\eval{\dv{s}}_{s = 0}\pulltimeflow{s}\alpha = \pulltimeflow{t_0}\ld[X]\alpha.
 			\end{align*}
 			Which was what we wanted.
 		\end{proof}
	
	\subsubsection{Cartan's magic formula}
		The combination of these three operators comes in the form of Cartan's magic formula. It tells us that the Lie derivative can be expressed in terms of the interior multiplication and exterior derivative. Concretely this is stated as follows.
		\begin{proposition}\label{prp: cartan magic formula}
			The Lie derivative of along a vector field \(X\) is given by
			\begin{equation*}
				\ld[X] = d\circ\iota_X + \iota_X\circ d.
			\end{equation*}
		\end{proposition}
		Before we prove this statement, we will go into some properties of the \textbf{commutator of \(d\) and \(\iota_X\)}, given by \(D_X = d\circ\iota_X + \iota_X\circ d\).
		\begin{lemma}\label{lem: action commutator}
			For a vector field \(X\in\vf\) and function \(f\in \sff\), the action of \(D_X\) is equal to the Lie derivative.
		\end{lemma}
		\begin{proof}
			Let \(X\) be a vector field on \(\m\) and \(f\) a function. By using the definition of the interior multiplication, Proposition~\ref{part: exterior derivative differential}, and Corollary~\ref{cor: lie derivative is action}, it follows that 
			\begin{equation*}
				D_Xf = \h{d\circ\iota_X + \iota_X\circ d}f = d\h{\iota_Xf} + \iota_X\h{df} = df\h{X} = Xf = \ld[X]f.
			\end{equation*}
			Hence, this shows that \(D_Xf = \ld[X]f\).
		\end{proof}
		\begin{lemma}\label{lem: commutator commutes}
			The operator \(D_X\) commutes with \(d\), i.e. \(D_X\circ d = d\circ D_X\).
		\end{lemma}
		\begin{proof}
			This proof simply follows from the fact that the exterior derivative's square is zero, see Proposition~\ref{part: exterior derivative square}. We can see that
			\begin{equation*}
				D_X\circ d = \h{d\circ\iota_X + \iota_X\circ d}\circ d = d\circ\iota_X\circ d = d\circ\h{d\circ\iota_X + \iota_X\circ d} = d\circ D_X.
			\end{equation*}
			Therefore, \(d\) and \(D_X\) commute.
		\end{proof}
		\begin{lemma}\label{lem: commutator derivative}
			For a vector field \(X\in\vf\), the operator \(D_X\) is a derivation on \(\Omega\h{\m}\) of degree \(0\).
		\end{lemma}
		\begin{proof}
			Suppose that \(\m\) is a manifold, \(X\in\vf\). We can easily verify that \(D_X:\Omega^k\h{\m}\to\Omega^k\h{\m}\) which implies it has degree \(0\).
			
			To prove that it is a derivation, we will make use of Propositions~\ref{prp: linear contraction anti-derivation} and~\ref{part: exterior derivative anti-derivation}. Suppose that \(\omega\in\Omega^k\h{\m}\) and \(\eta\in\Omega^l\h{\m}\), then
			\begin{align*}
				D_X\h{\omega\wedge\eta}
				&= \h{d\circ\iota_X + \iota_X\circ d}\h{\omega\wedge\eta}\\
				&= d\h{\iota_X\omega\wedge\eta + \h{-1}^k\omega\wedge\iota_X\eta} + \iota_X\h{d\omega\wedge\eta + \h{-1}^k\omega\wedge d\eta}\\
				\notag&= d\iota_X\omega\wedge\eta + \h{-1}^{k - 1}\iota_X\omega\wedge d\eta + \h{-1}^kd\omega\wedge\iota_X\eta + \h{-1}^{2k}\omega\wedge d\iota_X\eta\\
				&\qquad + \iota_Xd\omega\wedge\eta + \h{-1}^{k + 1}d\omega\wedge\iota_X\eta + \h{-1}^k\iota_X\omega\wedge d\eta + \h{-1}^{2k}\omega\wedge\iota_Xd\eta\\
				&= \h{d\circ\iota_X + \iota_X\circ d}\omega\wedge\eta + \omega\wedge\h{d\circ\iota_X + \iota_X\circ d}\eta = D_X\omega\wedge\eta + \omega\wedge D_X\eta.
			\end{align*}
			This shows that, \(D_X\h{\omega\wedge\eta} = D_X\omega\wedge\eta + \omega\wedge D_X\eta\) and it is therefore a derivation of degree \(0\).
		\end{proof}
		We will now apply Lemmas~\ref{lem: action commutator} and~\ref{lem: commutator derivative} to prove Proposition~\ref{prp: cartan magic formula}.
		\begin{proof}[Proof of Proposition~\ref{prp: cartan magic formula}]
			As all the operators satisfy the product rule, we can deduce that they are local, see the proof of Proposition~\ref{prp: existence exterior derivative manifold}. If we combine this with the linearity of the operators, we remark that we only need to consider a \(k\)-form \(\omega\) which we can write in a coordinate chart \(\h{U,\h{x^i}}\) as \(\omega|_U = \omega_{i_1,\ldots, i_k}dx^{i_1}\wedge\cdots\wedge dx^{i_k}\). Then we can split this \(k\)-form into two parts \(\omega|_U = \alpha\wedge\eta\) with \(\alpha = dx^{i_1}\) and \(\beta = \omega_{i_1\ldots i_k}dx^{i_1}\wedge\cdots\wedge dx^{i_k}\). Then remark that \(\omega\) is some wedge product of an exact \(1\)-form and a \(k - 1\)-form. By using the derivation properties of the commutator and the Lie derivative, see Lemma~\ref{lem: commutator derivative} and Proposition~\ref{prp: lie derivative derivation}, we can show the equivalence on a \(k\)-form by proving the equivalence on exact \(1\)-forms and using induction. To prove the equivalence on exact \(1\)-form, we will use Lemmas~\ref{lem: action commutator} and~\ref{lem: commutator commutes} and Proposition~\ref{prp: exterior derivative and lie derivative commute}.
			\begin{equation*}
				D_X\h{df} = d\h{D_Xf} = d\h{\ld[X]f} = \ld[X]\h{df}.
			\end{equation*}
			This is then enough to prove Cartan's magic formula as we know that they coincide on \(0\)-forms, see Lemma~\ref{lem: action commutator}.
		\end{proof}
	
	\subsubsection{Poincaré's lemma}
		Let us now apply Cartan's magic formula to prove a basic statement on differential forms called Poincaré's lemma. This tells us that any closed differential form is locally also exact. We will first prove this for differential forms on open subsets of \(\R[n]\), which naturally extends to the mentioned statement. Our proof consists of defining an operator that almost inverts the exterior derivative, called a homotopy operator.
		\begin{lemma}
			Every closed \(k\)-form with \(k\geq 1\) on a star-shaped domain of \(\R[n]\) is exact.
		\end{lemma}
		\begin{proof}
			Suppose that \(V\) is a star-shaped domain of \(\R[n]\) and let \(\omega\)	be a \(k\)-form on \(V\) with \(k \geq 1\). Without loss of generality, we can	assume that \(V\) is star-shaped around \(0\). We want to construct an operator \(h:\Omega^k\h{V}\to\Omega^{k - 1}\h{V}\) such that \(d\circ h = \id\). This is impossible in general as it would imply that every differential form is exact. Therefore we generalize our formula to \(d\circ h + h\circ d = \id\), such that it is equivalent to \(d\circ h = \id\) for closed forms.
			
			This operator \(h\) is defined using the contraction mapping \(m_t:V\to V:x\mapsto tx\), which is a well-defined function as long as \(t\in\ha{0,1}\) due to the star-shapedness of \(V\). Remark that \(m_0 = 0\) and \(m_1 = \id_V\), therefore \(\pull{m_0} = 0\) and \(\pull{m_1} = \id_{\Omega\h{V}}\) as well. Define \(h:\Omega^k\h{V}\to\Omega^{k - 1}\h{V}\) as follows
			\begin{equation*}
				h\h{\omega} = \int_{0}^1\dfrac{1}{t}\pull{m_t}\h{\iota_X\omega}dt.
			\end{equation*}
			where \(X = x^i\pdv*{x^i}\) such that \(m_t = \phi_X^{\ln\h{t}}\). Remark that this integral is well-defined even though we divide by \(t\). This can be seen by writing the integrand out in coordinates, where we assume \(\omega = \omega_{i_1\ldots i_k} dx^{i_1}\wedge\cdots\wedge dx^{i_k}\) in this calculation as every operation is linear. By noticing that \(m_t\) is simply the multiplication by \(t\) in each coordinate and the fact that \(d\) is \(\R\) linear, we can calculate the action of the pullback in coordinates, see Lemma 14.16 (c) in \cite{Lee2013}, we see
			\begin{align*}
				&\hspace{-1mm}\dfrac{1}{t}{\pull{m_t}\h{\iota_X\omega}}\\
				&= \dfrac{1}{t}\pull{m_t}\h{\iota_X\omega_{i_1\ldots i_k}\ dx^{i_1}\wedge\cdots\wedge dx^{i_k}}\\
				&= \dfrac{1}{t}\pull{m_t}\h{\sum_{j = 1}^k\h{-1}^{j - 1}dx^{i_j}\h{X}\omega_{i_1\ldots i_k} dx^{i_1}\wedge\cdots\wedge dx^{i_{j - 1}}\wedge dx^{i_{j + }}\wedge\cdots\wedge dx^{i_k}}\\
				&= \dfrac{1}{t}\sum_{j = 1}^k\h{-1}^{j - 1}X\h{tx^{i_j}}\h{\omega_{i_1\ldots i_k}\circ m_t}d\h{tx^{i_1}}\wedge\cdots\wedge d\h{tx^{i_{j- 1}}}\wedge d\h{tx^{i_{j + 1}}}\wedge\cdots\wedge d\h{tx^{i_k}}\\
				&= t^{k - 1}\sum_{j = 1}^k\h{-1}^{j - 1}X\h{x^{i_j}}\h{\omega_{i_1\ldots i_k}\circ m_t}dx^{i_1}\wedge\cdots\wedge dx^{i_{j - 1}}\wedge dx^{i_{j + 1}}\wedge\cdots\wedge dx^{i_k}.\\
			\end{align*}
			This shows that the integrand is well-defined for \(t\in\ha{0,1}\) if \(k \geq 1\).
			
			All that is left to show, is that this operator satisfies the equation
			\begin{equation*}
				d\circ h + h\circ d = \id.
			\end{equation*}
			We can prove this by working out the calculation for some \(\omega\). In this calculation, we will use the fact that the integral is a linear operator and that the exterior derivative commutes with the integral and the pullback, see Propositions~\ref{prp: exterior derivative integral} and~\ref{prp: exterior derivative pullback}
			\begin{align*}
				\h{d\circ h + h\circ d}\omega
				&= d\int_{0}^1\dfrac{1}{t}\pull{m_t}\h{\iota_X\omega}dt + h\h{d\omega},\\
				&= \int_{0}^1\dfrac{1}{t}\pull{m_t}\h{d\h{\iota_X\omega}}dt + \int_{0}^1\dfrac{1}{t}\pull{m_t}\h{\iota_X\h{d\omega}}dt,\\
				&= \int_{0}^1\dfrac{1}{t}\pull{m_t}\h{\h{d\circ\iota_X + \iota_X\circ d}\omega}dt.\\
				\intertext{
					Now we use Cartan's magic formula to rewrite this in terms of the Lie derivative. Furthermore, we recognise \(m_t\) as the flow of \(X\) and use the commutation property of the Lie derivative. Then we obtain our results from the chain rule.
				}
				\h{d\circ h + h\circ d}\omega
				&= \int_{0}^1\dfrac{1}{t}\pull{m_t}\h{\ld[X]\omega}dt = \int_{0}^1\dfrac{1}{t}\pull{\h{\phi_X^{\ln\h{t}}}}\h{\ld[X]\omega}dt,\\
				&= \int_{0}^1\dfrac{1}{t}\eval{\dv{s}}_{s = \ln\h{t}}\h{\pull{\h{\phi_X^s}}\omega}dt,\\
				&= \int_0^1\dv{t}\h{\pull{\h{\phi_X^{\ln\h{t}}}}\omega}dt = \int_0^1\dv{t}\h{\pull{m_t}\omega}dt = \pull{m_1}\omega - \pull{m_0}\omega = \omega.
			\end{align*}
			Hence, the operator \(h\) as defined above is exactly the operator we were looking for. By using the fact that \(\omega\) is closed, we can deduce that \(d\h{h\h{\omega}} = \omega\) and thus \(\omega\) is exact on \(V\).
		\end{proof}
		\begin{corollary}\label{cor: closed is exact}
			For every closed differential form \(\omega\) and point \(p\) on the manifold, there exists a neighbourhood of \(p\) on which \(\omega\) is exact.
		\end{corollary}
		\begin{proof}
			Let \(\m\) be a \(n\)-manifold and suppose that \(\omega\) is a closed \(k\)-form on \(\m\) and let \(p\) be a point in \(\m\). Take a chart \(\h{U,\phi}\), such that \(\phi:U\to V\subset\R[n]\) is a diffeomorphism. We can assume that \(V\) is star-shaped because if it is not we can take an open sphere \(\tilde{V}\) contained in \(V\) as \(V\) is open, the diffeomorphism \(\phi|_{\phi^{-1}\h{\tilde{V}}}\) then gives a new chart.
			
			We would like to use Poincaré's lemma now, however, this lemma only considers differential forms on \(\R[n]\). Therefore, we want to use the diffeomorphism of the chart to pullback the differential form to a differential form on \(V\).
			
			To be able to do this use that the exterior derivative commutes with the pullback such that \(\pull{\h{\phi^{-1}}}\eval{\omega}_U\) a closed \(k\)-form on \(V\). As we assumed that \(V\) is star-shaped we can use Poincaré's lemma implying the existence of a \(\h{k - 1}\)-form \(\eta\) on \(V\) with the property that \(\pull{\h{\phi^{-1}}}\eval{\omega}_U = d\eta\). By again taking the pullback by \(\phi\), and using the fact that \(\pull{\h{\phi^{-1}\circ\phi}} = \id_U\), it follows that
			\begin{equation*}
				\omega\big|_U = \pull{\h{\phi^{-1}\circ\phi}}\omega\big|_U = \pull{\phi}\h{\pull{\h{\phi^{-1}}}\omega\big|_U} = \pull{\phi}\h{d\eta} = d\h{\pull{\phi}\eta}.
			\end{equation*}
			This implies that \(\omega\) is exact on \(U\).
		\end{proof}
\end{document}


	\subsection{Darboux's Theorem}\label{sec: darboux}
		We are now able to prove Theorem~\ref{thm: darboux}. However, much like the Riemannian metric, we can also give a short intuitive proof of this statement akin to Riemann's counting argument given above. It now deviates as we can use Corollary~\ref{cor: closed is exact}. When given a symplectic form \(\omega\) it can be written is coordinates \(\h{x^i}\) as \(\omega = d\theta = d\h{\theta_idx^i}\). This then leaves just \(n\) function we need to determine, however with our \(n\) choices for coordinates we have these exactly covered and there are no local invariants for a symplectic form.
		
		Now we will give a formal proof, for which will restate the theorem once more. Then we will go into the proof where we use linear symplectic geometry and extend its structure locally using Moser's trick, Poincaré's lemma and Cartan's magic formula. 
		\darboux*
		\begin{proof}
			To prove this theorem for a point \(p\) on a symplectic manifold \(\h{\m,\omega}\) we will do the following. Firstly, we will use Theorem~\ref{thm: existence symplectic basis} to generate a basis for the tangent space at this point. We will then extend this basis to a neighbourhood of this point, generating a chart. Subsequently, we will show that part of this neighbourhood can be pulled back to the standard symplectic form in these coordinates. This will then define coordinates that satisfy Darboux's theorem.
			
			Suppose that \(\h{\m,\omega}\) is a symplectic \(m\)-manifold and \(p\in\m\). By theorem~\ref{thm: existence symplectic basis} there exists a symplectic basis of \(\loctang{p}\), denoted as \(\symbasv{u}{v}\). This also implies that \(m = 2n\) such that \(\m\) is even dimensional. Furthermore, the symplectic form \(\omega_p\) can be written in the associated dual basis \(\symcorv{\mu}{\nu}\) as
			\begin{equation*}
				\omega_p = \sum_{i = 1}^n \mu^i\wedge \nu^i.
			\end{equation*}
			We will try to extend this basis to a coordinate chart \(\symcor{x}{y}\) such that \(\mu^i = dx^i\) and \(\nu^i = dy^i\). These can be found rather easily by taking an arbitrary coordinate chart around \(p\), \(\h{U,\symcor{\tilde{x}}{\tilde{y}}}\), and then remarking that both \(\symcor{\mu}{\nu}\) and \(\symcor[p]{d\tilde{x}}{d\tilde{y}}\) are both basis for \(\loccotang{p}\). Hence, there exists a basis transformation, which is linear and non-singular, given by some matrices \(A,B,C\) and \(D\)
			\begin{equation*}
				\mu^i = A^i_jd\tilde{x}^j_p + B^i_jd\tilde{y}^j_p,\quad\nu^i = C^i_jd\tilde{x}^j_p + D^i_jd\tilde{y}^j_p.
			\end{equation*}
			Then define now coordinates \(\symcor{x}{y}\) as follows
			\begin{equation*}
				x^i = A^i_j\tilde{x}^j + B^i_j\tilde{y}^j,\quad y^i = C^i_j\tilde{x}^j + D^i_j\tilde{y}^j.
			\end{equation*}
			These form a coordinate chart, as they are the composition of a diffeomorphism and a non-singular linear map. Furthermore, notice that by definition \(\mu^i = dx^i_p\) and \(\nu^i = dy^i_p\). Thus we have found a coordinate chart \(\h{U,\phi = \symcor{x}{y}}\) that satisfies
			\begin{equation*}
				\omega_p = \sum_{i = 1}^ndx^i_p\wedge dy^i_p.
			\end{equation*}
			We now want to compare \(\omega\) with the standard symplectic form \(\omega_1 = \sum_{i = 1}^ndx^i\wedge dy^i\). To do this, we will have to restrict to some smaller neighbourhood of \(p\) than \(U\). We will construct a symplectomorphism \(\psi\) between \(\h{V,\omega}\) and \(\h{V,\omega_1}\) such that \(\psi\h{p} = p\). Let us set \(\omega_0 = \omega\) in this process.
			
			To construct such a symplectomorphism we use Moser's trick. This comes down to constructing an isotopy, which is a map \(\rho:U\times\ha{0,1}\to U:\h{p,t}\mapsto\rho_t\h{p}\) such that \(\rho_t\) is a diffeomorphism on \(U\) and \(\rho_0 = \id_{U}\). We will give a short exposition to motivate this construction. Suppose that we are given an isotopy \(\rho_t\) and a family of symplectic forms \(\omega_t\) such that \(\pull{\rho_t}\omega_t = \omega_0\). Define some vector field \(X_t\) generated by \(\rho_t\) as
			\begin{equation*}
				X_t = \dv{\rho_t}{t}\circ\rho_t^{-1}.
			\end{equation*}
			Let us now consider the derivative of the pullback of \(\omega_t\) by \(\rho_t\). We can rewrite this using Proposition \ref{prp: lie derivative flow commute} and Cartan's magic formula.
			\begin{equation*}
				0 = \eval{\dv{s}}_{s = t}\h{\pull{\rho_s}\omega_s} = \pull{\rho_t}\h{\ld[X_t]\omega_t + \dv{\omega_t}{t}} = \pull{\rho_t}\h{d\h{\iota_{X_t}\omega_t} + \dv{\omega_t}{t}}.
			\end{equation*}
			As \(\rho_t\) is a diffeomorphism, we deduce the following equation called Moser's equation:
			\begin{equation*}
				\h{d\circ\iota_{X_t}}\omega_t = -\dv{\omega_t}{t}.
			\end{equation*}
			Hence, an isotopy generates a vector field which satisfies Moser's equation. Remark that we could also recover the isotopy if we were given a vector field that satisfies Moser's equation. Hence, to generate an isotopy, we will try to solve Moser's equation.
			
			Consider the difference \(\eta = \omega_1 - \omega_0\) and remark that this is a closed \(2\)-form. Therefore, by Corollary~\ref{cor: closed is exact} we can find a neighbourhood \(U_0\) of \(p\) such that \(\eta\) is exact, i.e. there exists a \(1\)-form \(\alpha\) such that \(\eta = -d\alpha\) and we can assume that \(\alpha_{p}= 0\) using the linearity of the exterior derivative. Furthermore, we can assume that \(U_0\subset U\). We will then consider the family of symplectic forms \(\omega_t\) with \(t\in\ha{0,1}\) given by
			\begin{equation*}
				\omega_t = \omega_0 - td\alpha = \h{1 - t}\omega_0 + t\omega_1.
			\end{equation*}
			This is again closed and as \(\omega_t|_{p}\) is non-degenerate and \(\omega_t\) is smooth it follows that \(\omega_t\) is non-degenerate in a neighbourhood \(U_1\) of \(p\). Again, we can assume that \(U_1\subset U_0\). Thus we have found a family of symplectic forms that we can enter into Moser's equations to get the following
			\begin{equation*}
				\h{d\circ\iota_{X_t}}\omega_t = -\dv{\omega_t}{t} = \omega_0 - \omega_1 = -\eta = d\alpha.
			\end{equation*}
			To solve this equation, it is sufficient to solve for \(\iota_{X_t}\omega_t = \alpha\). By the non-degeneracy of \(\omega_t\), this equation can be solved locally. As for an arbitrary \(Y\in\vf\) we can calculate this expression in the coordinates \(\h{z^1,\ldots,z^n,z^{n + 1},\ldots,z^{2n}} = \h{x^1,\ldots,x^n,y^1,\ldots,y^n}\). The left-hand side then becomes
			\begin{align}
				\iota_{X_t}\omega_t\h{Y} 
				\notag&= \omega_t\h{X_t,Y} = \sum_{i,j}\dfrac{1}{2}\h{\omega_t}_{ij}dz^i\wedge dz^j\h{X_t,Y}
				= \sum_{i,j}\dfrac{1}{2}\h{\omega_t}_{ij}\h{X_t^iY^j - X_t^jY^i},\\
				\label{eq: lefthandside}&= \sum_{i,j}\dfrac{1}{2}\h{\h{\omega_t}_{ij} - \h{\omega_t}_{ji}}X_t^iY^j =
				\sum_{i,j}\h{\omega_t}_{ij}X_t^iY^j.
			\end{align}
			Meanwhile, the right-hand side can be expressed as
			\begin{equation}
				\label{eq: righthandside}\alpha\h{Y} = \sum_{j}\alpha_j dz^j\h{Y} = \sum_{j}\alpha_jY^j.
			\end{equation}
			As \(Y\) is arbitrary, we can assume that there is some \(1\leq k\leq 2n\) such that \(Y = \pdv*{z^k}\). Combining Equations \ref{eq: lefthandside} and \ref{eq: righthandside} then gives us that \(\sum_{i}\h{\omega_t}_{ik}X_t^i = \alpha_k\). As \(\omega_t\) is non-degenerate, \(\h{\omega_t}_{ik}\) has an inverse, which we denote as \(\h{\omega_t}^{ik}\). We can then express \(X_t\) in terms of \(\omega_t\) and \(\alpha\)
			\begin{equation*}
				X_t^i = \sum_{k}\h{\omega_t}^{ik}\alpha_k.
			\end{equation*}
			From the smoothness of \(\omega_t\) and \(\alpha\), and the fact that \(\h{\omega_t}^{ik}\) is a rational function of the coefficients of \(\h{\omega_t}_{ik}\) it follows that \(X_t^i\)  is smooth as well. Thus we can smoothly solve for \(X_t\).
			
			Remark that \(\alpha_{p} = 0\) and therefore \(X\h{t,p} = 0\) which implies that \(\phi_X^{t,0}\h{p} = p\) for all \(t\in\ha{0,1}\). As the flow domain of is open, we can use the tube lemma, see \cite[Lemma 26.8]{Munkres2013} to find a neighbourhood \(U_2\) of \(p\) such that \(\phi_X^{t,0}\) is defined on \(U_2\) for all \(t\in\ha{0,1}\), where we can once again assume \(U_2\subset U_1\). Then by defining \(\rho_t = \phi_X^{t,0}\), we get the isotopy we wanted as \(\rho_0 = \phi_X^{0,0} = \id_{U_2}\), and we have already shown that this satisfies Moser's equation. It follows that \(\psi = \rho_1\) defines a symplectomorphism on \(U_2\) between \(\omega_0\) and \(\omega_1\) that preserves \(p\).
			
			We can now define the coordinate chart on \(V = U_2\) by \(\hat{x}^i = x^i\circ \psi\) and \(\hat{y}^i = y^i\circ\psi\), then it follows
			\begin{equation*}
				\sum_{i = 1}^n d\hat{x}^i\wedge d\hat{y}^i = \sum_{i = 1}^nd\h{x^i\circ\psi}\wedge d\h{y^0\circ\psi} = \pull{\psi}\h{\sum_{i = 1}^n dx^i\wedge dy^i} = \pull{\psi}\omega_1 = \omega_0.
			\end{equation*}
			Thus \(\h{\hat{x}^i, \hat{y}^i}\) are the coordinates we wanted.
		\end{proof}
		\begin{remark}
			The closedness of the symplectic form is necessary as \(\omega_{\st,2n}\) is closed. If \(\omega\) is symplectomorphic to \(\omega_{\st,2n}\) in some neighbourhood, this symplectomorphism would preserve the closedness. Hence, closedness is necessary for the symplectic form to be `flat'.
		\end{remark}
		The local theory is therefore always that of the trivial theory, however, this does not mean that there are no interesting global phenomena to explore. We will not go in this direction now, instead, we will discuss one of the main applications and the birthplace of symplectic geometry: Classical mechanics.
\end{document}
