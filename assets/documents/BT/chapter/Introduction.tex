\documentclass[class = article, crop = false]{standalone}
\begin{document}
\chapter{Introduction}
Any avid student of both mathematics and physics has come in contact with a manifold at some point. The apt description of a space like the earth, a sphere which looks flat for us standing on it, immediately paints the usefulness of an abstract object like a manifold in the description of the real physical world. It is in this marriage of mathematics and physics that symplectic geometry is born and where it was first found as a manifold with a geometric structure much like a Riemannian manifold. Lagrange is known as the first who came upon these structures in his research on constants of motion in \(1808\). It was first known as complex groups due to their association with line complexes, but the inverter of this name, H. Weyl, noted that this name seemed to imply a connection to complex numbers. He proposed to use the Greek calque: symplectic, which is what we use nowadays. Even then, symplectic geometry only played a role in the research of mathematical physics and it was not until V.I. Arnol’d posed some interesting conjectures surrounding symplectic geometry in \(1960\) that it became a separate field in mathematics.

In this thesis, we will focus our attention on the local structure of symplectic manifolds and Hamiltonian systems. Therefore, it is of utmost importance that the reader is familiar with the concepts surrounding smooth manifolds. There is a heavy focus on tensor fields and differential forms, hence, one should at least be comfortable with their construction and the definition of the exterior derivative. Furthermore, as many different authors have their own notations for vector fields and flows, we have included an appendix discussing vector fields, flows and time-dependent vector fields where all the notation is also defined. Good sources on all the prerequisites are Chapters 1, 2, 3, 4, 8, 9, 12 and 14 \cite{Lee2013} or the first 13 lectures of \cite{Marcut2017} excluding Lecture 11. These sources are also used for most proofs in manifold theory. As these sources do not contain adequate chapters on symplectic geometry, we will follow \cite{CannasdaSilva2008} for symplectic geometry and Hamiltonian systems, and \cite{Arnold1989} for Hamiltonian systems. We will not assume any prior knowledge on the topic of symplectic geometry, be it on vector spaces or manifolds, and all necessary definitions and references are included in this thesis. At some points, we do assume the reader is familiar with Riemannian geometry. However, these parts are never mandatory to understand the results of symplectic geometry as they serve the purpose of adding context to our results. Lastly, we work out a couple of examples of both mechanical and Hamiltonian systems, including some figures of mechanical motions made with TikZ and Python.

\subsubsection{Motivation}
As mentioned in the introduction, any student of both the mathematics track and the physics track comes in contact with manifolds at some point, as did I. In the previous year, I followed a triplet of courses introducing the concepts of smooth manifolds, Riemannian geometry and mathematical physics. It was in this last course that I was introduced to symplectic geometry as a method in mathematical physics as a geometric method to classical mechanics. Meanwhile, in the course on Riemannian geometry, we were studying a close relative to symplectic geometry. Together this piqued my interest in geometry, more specifically symplectic geometry. Together with my supervisor, Ioan M\u{a}rcu\textcommabelow{t}, I proposed to investigate the basic geometry of a symplectic manifold and then showcase how this geometry comes together with physics. This marriage of physics and mathematics is what inspired me to choose this topic.

Besides this emotional note of motivation for this thesis, it of course also has a lot of mathematical merits. The flatness of all symplectic manifolds paints a stark contrast when put next to its not-so-flat cousin Riemannian geometry. Furthermore, the study of Hamiltonian systems, a simple variation on the structure of symplectic manifolds, let us find many useful results about classical mechanics using a geometric point of view. Furthermore, this thesis gives an introduction to the topic of symplectic geometry and can be used as a starting point for further study.

\subsubsection{Conventions}
All notations and conventions on manifold theory are taken from \cite{Lee2013}, but as we already mentioned any vector field-related quantities are defined in Appendix~\ref{app: vector fields}. We will follow the conventions and notation of \cite{CannasdaSilva2008} for symplectic geometry as it forms the main source on this topic. Furthermore, we assume that every manifold and vector field in this thesis is smooth, we usually only mention the smoothness in the context of functions.

\subsubsection{Acknowledgements}
I would like to thank my supervisor, Ioan M\u{a}rcu\textcommabelow{t}, for all the help and patience even when on sabbatical. Thank you for all the explanations and the interesting working environment, allowing me to find a topic which interested me. Furthermore, I would like to thank Klaas Landsman for being my second reader. Secondly, I want to thank Rozemarijn for her feedback but more importantly her support and ever-lasting willingness to be a listening ear. Furthermore, many thanks to my friends and roommates for their joining me in my writing sessions. Lastly, I want to thank my family for always motivating me and giving me the confidence to go on.
\end{document}