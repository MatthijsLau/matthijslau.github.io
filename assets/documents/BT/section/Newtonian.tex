\documentclass[class = article, crop = false]{standalone}
\usepackage{standalone}

\begin{document}
	\section{Newtonian Formalism}
	We will make our first step into describing classical mechanics by taking a look at Newton's formulation and basic principles in the form of his three laws of motion. Nowadays these still form the foundation of classical mechanics. Newton was one of the first persons who saw that we could describe physics using some general mathematical model, which can be seen as the goal of classical physics nowadays: to predict the motions of a physical space using a mathematical model. Before we go into the actual model Newton built, we will discuss how we can even translate a physical space to a mathematical one in the first place.
	
	\subsection{Space, Time and Kinematics}
	As the goal is to describe the motions of mechanical systems mathematically, we should first determine the types of systems we are studying and how we could translate these to mathematical spaces. The assumptions in classical mechanics are that the objects are relatively large such that there are no quantum mechanical effects and the speeds are relatively small to stay non-relativistic. We then follow our intuition and suppose that positions in physical space are points of a three-dimensional Euclidean space \(E^3\). We would like to induce some vector space structure onto \(E^3\), which can be done by fixing an origin \(o\in E^3\), also called an observer, and we then identify a point \(s\in E^3\) with the vector \(\vec{os}\in\R^3\). When working with these vectors, we then also need to choose some basis vectors. We have some freedom of choice for what position acts as the origin and how we arrange the basis vectors, depending on this choice the position of an object may seemingly change, see Example~\ref{exp: reference frame plane}. In a physical problem, we can often choose a reference frame that uses the symmetries of a system.
	\begin{example}\label{exp: reference frame plane}
		Let us assume that the reference frames are chosen such that the positions in this example are constrained to \(\R[2]\times\hv{0}\subset\R[3]\). Therefore, we will only consider the position on a plane instead of a three-dimensional space. Suppose that we have a ball at a point \(p\) on \(E^3\) and some observers \(A\) and \(B\), see Figure~\ref{fig: configuration}. We can then view the position \(p\) from both the reference frame of \(A\) and \(B\). We can then measure the position of \(P\) in both reference frames, see Figure~\ref{fig: measurements}. Observer \(A\) measures the position of \(P\) as \(r_{PA} = \h{4,-2}^T\), while Observer \(B\) measures \(P\) to be at \(r_{PB} = \h{-1,2}^T\).
	\end{example}
	\begin{figure}
		\centering
		\begin{subfigure}[t]{.49\textwidth}
			\centering
			\includegraphics{img/ReferenceFrame_Configuration.pdf}
			\caption{The space \(E^3\) given by some grid on a plane on which a ball is placed at position \(P\) and two observers are positioned at \(A\) and \(B\).}
			\label{fig: configuration}
		\end{subfigure}
		\hfill
		\begin{subfigure}[t]{.49\textwidth}
			\centering
			\includegraphics{img/ReferenceFrame_Measurement.pdf}
			\caption{The reference frames and measurements of observer \(A\) and \(B\). The reference frame of observer \(A\) is given in blue, and the measured position of \(P\) is denoted by \(r_{PA}\). For observer \(B\) everything is in red and the position is given by \(r_{PB}\).}
			\label{fig: measurements}
		\end{subfigure}
		\caption{Example of how to translate a physical situation, Figure~\ref{fig: configuration}, to measurements in different reference frames, Figure~\ref{fig: measurements}.}
		\label{fig: numberline}
	\end{figure}
	As we mentioned, we are not dealing with any relativistic effects in this theory, and hence, we can identify time as a separate axis, called the time axis identifiable with \(\R\). This time axis is important when talking about motion, as this is all about the position over time. We define the motion of an object as some smooth function \(\mathcal{C}:I\to E^3\), where \(I\) is an interval on the time axis. In our reference frame, we can obtain a function \(r:I\to\R[3]\) called the trajectory of the object. Notice that this is the composition of the actual motion with our choice of reference frame, hence, it is dependent on this choice which may differ over time. Using this mathematical trajectory, we can describe some physical quantities as vectors. Namely, the velocity and acceleration, \(v\) and \(a\) respectively, are defined as follows
	\begin{equation*}
		v\h{t} = \dv{t}r\h{t} = \dot{r}\h{t}\qquad\mbox{and}\qquad a\h{t} = \dv[2]{t}r\h{t} = \ddot{r}\h{t} = \dot{v}\h{t}.
	\end{equation*}
	The importance of these quantities stems from Newton's laws of motion. However, before discussing these in detail, we should find a way to model more than just a single object in a system. We have thus far described a position of a single object while we are often dealing with a system that includes many bodies that are interacting. Intuitively we assign a single \(E^3\) for each object in the system, which can be formalised in terms of a product space. In other words, in an \(n\)-body system a position of the system is described as a point in \(E^{3n} = E^3\times\cdots\times E^3\). The position of the bodies in the system is described using a single vector \(r = \h{r_1,\ldots, r_n}\in\R[3n]\), implying that the motion becomes a smooth map \(r\h{t} = \h{r_1\h{t},\ldots,r_n\h{t}}\in\R[3n]\). The definition of velocity and acceleration then still applies.
	
	\subsection{Newton's Laws of Motion}\label{sec: laws of motion}
	Newton's laws of motion determine the motions of a system through two important concepts: momentum and forces. Momentum is a quantity of an object and is given for some object of mass \(m\) by \(p = mv\). This is often intuitively thought of as the amount of movement an object has. Forces on the other hand should be thought of as the interactions bodies have with each other or the system. These come in many shapes and forms, for example, the gravitational force or Coulomb force. All these forces are the product of some interaction between two physical bodies and are described as some vector \(F\in\R[3]\) acting on a body. As we will approximate motion to our best capability, it will often be useful to neglect some forces acting on larger bodies, this choice will make more sense in the light of Newton's second law. Let us now state these laws.
	\begin{enumerate}[label = {\arabic*.}]
		\item If the sum of all forces acting on a body is zero, there exists a reference frame in which its velocity is constant.
		\item In any inertial frame, the time derivative of the momentum of a body is equal to the sum of forces acting on it.
		\item For every action, there is an equal and opposite reaction.
	\end{enumerate}
	While the first law seems to follow from the second law, we still need to ensure the existence of an inertial frame to use the second law. The second law then gives us the following equation, also called the equation of motion
	\begin{equation*}
		\sum_iF_i\h{r,\dot{r},t} = \dot{p}\h{t},
	\end{equation*}
	where \(F_i\) are all the forces acting on the body, \(p\h{t}\) is its momentum at time \(t\). We will often denote the sum of all forces acting on a body with \(F_{\net}\). If we also assume that the body has a constant mass this equation simplifies to
	\begin{equation*}
		F_{\net}\h{t,r,\dot{r}} = m\ddot{r}.
	\end{equation*}
	If we generalise this to an \(n\)-body system, we obtain an equation of motion for each of the bodies. In the case that the masses of all objects are constant, which is nearly always the case, this results in the following system of equations:
	\begin{equation}\label{eq: n equation of motion}
		\mqty(\dot{r}\\\dot{v}) = \mqty(v\\M^{-1}F\h{r,v,t}),\qquad M = \diag\h{m_1I_3,\ldots,m_nI_3}.
	\end{equation}
	Remark that we have transformed the differential equation by substituting \(v = \dot{r}\). This system of equations is the one we solve most often in Newtonian mechanics. The process of solving for the motions of a system uses the following steps:
	\begin{enumerate}[label = {\alph*)}]
		\item Choose a reference frame, and initial values then determine which forces are at play.
		\item Describe the forces in terms of the reference frame.
		\item Solve the equation of motion.
	\end{enumerate}
	Let us showcase this with a couple of examples.
	\begin{example}\label{exp: projectile}
		\documentclass[class = article, crop = false]{standalone}
\usepackage{standalone}

\begin{document}
	\begin{figure}
		\centering
		\includegraphics{img/Projectile_Sketch.pdf}
		\caption{A sketch of a cannon atop a hill shooting a cannonball at some angle and speed. In the figure, we plotted a trajectory which we calculated by numerically solving Newton's equations of a projectile motion in a constant gravitational field with linear and quadratic air resistance using the SciPy package in Python. Our initial velocity for this trajectory was set to \(15\) ms\(^{-1}\) in the \(x\)-direction and \(3\) ms\(^{-s}\) in the \(y\)-direction and the height of the cliff was set at \(10\) m. We will see in our analysis that the mass of the cannonball is irrelevant to the problem. At the foot of the cliff, the reference frame for the analysis is drawn as well.}
		\label{fig: sketch projectile}
	\end{figure}
	Suppose we have set up a cannon atop an \(h\) meter high cliff and we want to determine the distance it can shoot a cannonball which weighs \(m\) kilograms, see Figure~\ref{fig: sketch projectile}. We would then want to find a mathematical model to capture the motion of the cannonball. 
	
	First, we need to determine a reference frame which we have already chosen in Figure~\ref{fig: sketch projectile} to be at the foot of the cliff with the \(x\) corresponding to the horizontal direction and the \(y\) axis with the vertical one. Next up, we determine the initial values of the system. From Equation~\ref{eq: n equation of motion}, it is clear that we need both the initial position and velocity of a system to solve for the motion. In this case, the initial position of the cannonball can be considered to be the position of the cannon, i.e. \(r\h{0} = \h{0,h}^T\). The initial velocities are determined by some parameters, such that \(v\h{0} = \h{v_{x0},v_{y0}}\). Lastly, we need to determine the forces which we simply have to guess and tune until we are satisfied with the model. In this case, we introduce just the force of gravity.
	\begin{equation*}
		F_{\st[grav]}\h{t,r,\dot{r}} = F_{\st[grav]} = \mqty(0\\-mg).
	\end{equation*}
	Here, \(g\in\R\) is the acceleration due to gravity. As this is the only force we will be considering, we need to solve the following initial value problem:
	\begin{equation*}
		\dot{u} = \mqty(\dot{r}_x\\\dot{r}_y\\\dot{v}_x\\\dot{v}_y) = \mqty(v_x\\v_y\\0\\-g) = \mqty(0&0&1&0\\0&0&0&1\\0&0&0&0\\0&0&0&0)u + \mqty(0\\0\\0\\-g),\qquad u\h{0} = \mqty(0\\h\\v_{x0}\\v_{y0}).
	\end{equation*}
	Remark that this equation is of the form \(\dot{u} = Au + b\), thus we can find the solution by integrating, see Theorem 2.4.1 in \cite{MyintU1978}, such that
	\begin{equation*}
		u\h{t} = e^{tA}u\h{0} + \int_0^te^{\h{t - s}A}bds = \mqty(v_{x0}t\\h + v_{y0}t - \flatfrac{gt^2}{2}\\v_{x0}\\v_{y0} - gt).
	\end{equation*}
	Therefore, we get the motion for the cannonball with gravity
	\begin{equation}\label{eq: projectile motion gravity}
		r\h{t} = \mqty(v_{x0}t\\h + v_{y0}t - \flatfrac{gt^2}{2}).
	\end{equation}
	\begin{figure}
		\centering
		\includegraphics{img/Projectile_Gravity.pdf}
		\caption{The trajectory of a cannonball as predicted by Equation~\ref{eq: projectile motion gravity} and the calculated trajectory of Figure~\ref{fig: sketch projectile}. The same initial conditions were used in this figure. Remark that the trajectories are rather close to each other, but we see that the model predicts the range of the cannon to be further than the calculated range. Yet, our model is quite close to the numerical analysis.}
		\label{fig: projectile grav}
	\end{figure}
	We have plotted an example of such a motion in Figure~\ref{fig: projectile grav}. Here we see that our model is not quite perfect, but it does approximate the trajectory quite well.
\end{document}
	\end{example}
	\begin{example}\label{exp: atwood newton}
		\documentclass[class = report, crop = false]{standalone}
\usepackage{standalone}

\begin{document}
	\begin{figure}
		\centering
		\begin{subfigure}[t]{.49\textwidth}
			\centering
			\includegraphics{img/Atwood_Sketch.pdf}
			\caption{The configuration of a pulley system with two masses of different sizes hanging from it.}
			\label{fig: atwood sketch}
		\end{subfigure}
		\hfill
		\begin{subfigure}[t]{.49\textwidth}
			\centering
			\includegraphics{img/Atwood_ForceDiagram.pdf}
			\caption{The Force diagram associated with Figure~\ref{fig: atwood sketch}. The reference frame is drawn at the centre of the pulley.}
			\label{fig: atwood force diagram}
		\end{subfigure}
		\caption{Pulley system as described in Example~\ref{exp: atwood newton} with both a sketch and a force diagram.}
		\label{fig: atwood}
	\end{figure}
	Let us consider an Atwood machine, which is a system of two stationary masses, \(m_1\) and \(m_2\), hanging on a rope over a pulley, see Figure~\ref{fig: atwood sketch}. Assume that the rope and pulley are massless, there is no friction in the pulley, the rope does not slip on the pulley and the length of the rope is constant. Set the origin at the centre of the pulley such that the initial positions of the masses are given by \(r_1 = \h{-R,-y_{10}}^T\) and \(r_2 = \h{R,-y_{20}}^T\), where \(R\) is the radius of the pulley.
	
	The forces in the system are then the force of gravity acting on both the objects, such that \(F_{\footm{grav},\footm{m}_i} = \h{0,-m_ig}^T\), and a tension force \(T = \h{0,T}^T\) which is the same on both objects by the constraint on the length of the rope. These forces are drawn in Figure~\ref{fig: atwood force diagram}. We can translate this to the following equation of motion.
	\begin{equation*}
		\mqty(m_1a_{x1}\\m_1a_{y1}\\m_2a_{x2}\\m_2a_{y2}) = \mqty(0\\T + F_{\footm{grav},\footm{m}_1}\\0\\T + F_{\footm{grav},\footm{m}_2}) = \mqty(0\\T - m_1g\\0\\T - m_2g).
	\end{equation*}
	We notice here that we can ignore the \(x\)-coordinates as there is no net force acting in this direction. However, when solving this equation, we run into the problem that \(T\) is an unknown. Luckily, we can recover it by remarking that \(a_1 = -a_2\), which is a result of the fact that the length of the rope is constant. We can solve this equation for \(T\).
	\begin{align*}
		\dfrac{T - m_1g}{m_1} = a_1 &= - a_2= -\dfrac{T - m_2g}{m_2}\\
		 m_2T - m_1m_2g &= -m_1T + m_1m_2g\\
		 T &= \dfrac{2m_1m_2}{m_1 + m_2}g.
	\end{align*}
	Substituting this into the equation of motion, we get
	\begin{equation*}
		\mqty(a_{y1}\\a_{y2}) = \mqty(\flatfrac{2m_2g}{\h{m_1 + m_2}} - g\\\flatfrac{2m_1g}{\h{m_1 + m_2}} - g) = \mqty(\flatfrac{g\h{m_2 - m_1}}{\h{m_1 + m_2}}\\\flatfrac{g\h{m_1 - m_2}}{\h{m_1 + m_2}}).
	\end{equation*}
	Solving this system is then rather simple and results in the motion of the masses.
	\begin{equation}\label{eq: atwood result newton}
		r\h{t} = \mqty(r_1\h{t}\\r_2\h{t}) = \mqty(y_{10} + \flatfrac{\mu gt^2}{2}\\y_{20} - \flatfrac{\mu gt^2}{2}),\qquad \mu = \dfrac{m_2 - m_1}{m_1 + m_2}.
	\end{equation}
	Here, we can see that the mass difference is the main factor in the dynamics. If the difference is zero, the masses will stay stationary. If one of them is heavier, the greater mass will move downwards.
\end{document}
	\end{example}
\end{document}