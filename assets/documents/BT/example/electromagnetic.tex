\documentclass[class = report, crop = false]{standalone}
\usepackage{standalone}

\begin{document}
	Let us consider the example of a particle with charge \(q\) moving through a transversal constant magnetic field and some electric potential. Due to the transversality of the magnetic field, we can consider the system in just two dimensions: the ones perpendicular to the direction of the magnetic field. Let us adopt the coordinates \(x\) and \(y\) to describe the positions of the system. Hence, our configuration space is given by \(\R[2]\) and the phase space by \(\h{\cotang[\R[2]],\omega_{\can}}\). We can derive from Example~\ref{exp: potential of lorentz force} that the potential energy can be expressed as
	\begin{equation*}
		U\h{x,y,\dot{x},\dot{y}} = V\h{x,y} - A\h{x,y}\dot{x} - B\h{x,y}\dot{y}.
	\end{equation*}
	In our case, we will choose the potential functions to be defined as
	\begin{equation}\label{eq: potentials EM}
		V\h{x,y} = qE\h{x^2 + y^2},\quad A\h{x,y} = \dfrac{My}{2},\quad B\h{x,y} = -\dfrac{Mx}{2}.
	\end{equation}
	This leads to the following Lagrangian for the system:
	\begin{equation*}
		\lag\h{x,y,\dot{x},\dot{y}} = \dfrac{1}{2}\h{\dot{x}^2 + \dot{y}^2} - qE\h{x^2 + y^2} + \dfrac{M}{2}y\dot{x} - \dfrac{M}{2}x\dot{y}.
	\end{equation*}
	From the Lagrangian, we obtain the generalised momenta associated with \(x\) and \(y\).
	\begin{equation*}
		p_x = \pdv{\lag}{\dot{x}} = \dot{x} + \dfrac{My}{2},\qquad p_y = \pdv{\lag}{\dot{y}} = \dot{y} - \dfrac{Mx}{2}.
	\end{equation*}
	Using the Legendre transform, we can deduce that the Hamiltonian is given by
	\begin{align*}
		\ham\h{x,y,p_x,p_y}
		&= \dot{x}p_x + \dot{y}p_y - \lag\h{x,y,\dot{x},\dot{y}},\\
		\notag&= p_x\h{p_x - \dfrac{My}{2}} + p_y\h{p_y + \dfrac{Mx}{2}} - \dfrac{1}{2}\h{p_x - \dfrac{My}{2}}^2 - \dfrac{1}{2}\h{p_y + \dfrac{Mx}{2}}^2\\
		&\qquad + qE\h{x^2 + y^2} - \dfrac{M}{2}y\h{p_x - \dfrac{My}{2}} + \dfrac{M}{2}x\h{p_y + \dfrac{Mx}{2}},\\
		&= \dfrac{1}{2}\h{\h{p_x - \dfrac{My}{2}}^2 + \h{p_y + \dfrac{Mx}{2}}^2} + qE\h{x^2 + y^2}.
	\end{align*}
	Remark that this is not the total energy, yet we recover the equations of motion from the Hamiltonian vector field
	\begin{equation}\label{eq: differential equations EM}
		X_{\ham}\h{d\alpha} =
		\begin{dcases}
			p_x - \dfrac{My}{2},&\alpha = x,\\
			p_y + \dfrac{Mx}{2},&\alpha = y,\\
			-\h{p_y + \dfrac{Mx}{2}}\dfrac{M}{2} - 2qEx,&\alpha = p_x,\\
			\h{p_x - \dfrac{My}{2}}\dfrac{M}{2} - 2qEy,&\alpha = p_y.
		\end{dcases}
	\end{equation}
	It is quite tricky to solve for the flow of this system analytically. Luckily, we can easily solve it numerically, see Figure~\ref{fig: numerical solution}
	\begin{figure}[t]
		\centering
		\includegraphics{img/ElectroMagneticTrajectory.pdf}
		\caption{A plot of the \(\h{x,y}\)-trajectory of a charged particle in an electric and magnetic field, generated by the potentials in Equation~\ref{eq: potentials EM}. The equations in Equation~\ref{eq: differential equations EM} were solved numerically using the SciPy package in Python, with the initial values of \(\h{x_0,y_0,p_{x0},p_{y0}} = \h{1,1,-2,2}\) and \(M = 0.1\), \(q = 0.5\) and \(E = 0.3\).}
		\label{fig: numerical solution}
	\end{figure}
\end{document}